\chapter{Resultaten}
\label{hoofdstuk:resultaten}
In dit hoofdstuk wordt het werk ingeleid. Het doel wordt gedefinieerd en er
wordt uitgelegd wat de te volgen weg is (beter bekend als de rode draad).

Als je niet goed weet wat een masterproef is, kan je altijd
Wikipedia eens nakijken.

\section{Praktische aspecten}
\section{Afhankelijkheid van aantal richtingen}
\begin{table}
    \centering
    \begin{tabular}{@{}lllllllll@{}} \toprule
      & \multicolumn{2}{c}{Feet} & \multicolumn{2}{c}{Sponza} & \multicolumn{2}{c}{Conference Hall} & \multicolumn{2}{c}{Museum} \\ \cmidrule(r){2-3} \cmidrule(r){4-5} \cmidrule(r){6-7} \cmidrule(r){8-9}
      
      K      & Mediaan     & Stdev & Mediaan & Stdev & Mediaan & Stdev & Mediaan & Stdev \\ \midrule
      4      &     & &     & & & & & \\
      5      &     & &     & & & & & \\
      6      &     & &     & & & & & \\
      7      &     & &     & & & & & \\
      8      &     & &     & & & & & \\
      9      &     & &     & & & & & \\
      10      &     & &     & & & & & \\ \bottomrule
    \end{tabular}
    \caption{Statistieken over de rendertijd voor $\symBSPrandomfastkd$ voor verschillende waarden van K. Voor elke waarde van K is het algoritme 6 keer uitgevoerd. }
    \label{tab:bsprandom-k-rendertijd}
  \end{table}
\section{Vergelijking}


%%% Local Variables: 
%%% mode: latex
%%% TeX-master: "masterproef"
%%% End: 