\chapter{Conclusie}
\label{hoofdstuk:conclusie}

\section{Conclusies $\symBSPsweep$}
De $\symBSPsweep$ boom is een zeer nuttig en uitbreidbaar concept om algemene $\symBSP$ bomen te bouwen.
De lokale geometrie van de knoop kan zeer makkelijk in rekening gebracht worden om goede splitsingsrichtingen te genereren.
Door het sweepen blijft de bouwtijd relatief beperkt in verhouding met de enige andere algemene $\symBSP$ boom: de $\symBSPize$ boom.
De bouwtijd is wel twee ordegroottes groter dan die van de $\symKd$ boom.
Voor tal van toepassingen is dit echter geen probleem aangezien de boom op voorhand één keer gebouwd kan worden.\\

De simpele $\symBSPsweep$ bomen die geen gebruik maken van de $\symKd$ richtingen, moeten bij normale scenes onderdoen voor de $\symKd$ boom en $\symBSPize$ bomen.
Bij de speciale Killeroo Been scene tonen deze bomen wel hun mogelijkheden, aangezien ze daarop beter werken dan de $\symKd$ boom.
De $\symBSPsweepkd$ bomen, die steeds de $\symKd$ richtingen nemen als eerste drie richtingen, presteren voor bepaalde waarden van $k$ (het aantal richtingen), beter dan de $\symKd$ en $\symBSPize$ bomen.
Deze bomen zullen in de praktijk echter nooit gebruikt worden omdat ze als opstapje dienen naar de $\symBSPsweepfastkd$ bomen, die altijd beter presteren dan de bestaande $\symBSP$ bomen.\\

De $\symBSPrandomfastkd$ boom toont dat het kiezen van andere splitsingsrichtingen in elke knoop, de boom veel beter maakt. 
Zelfs zonder rekening te houden met de lokale geometrie, kan dan een boom gebouwd worden die beter is dan de $\symKd$ boom of goed uitgedachte $\symBSPizefastkd$ boom. 
Dit komt voornamelijk door het grotere aantal verschillende splitsingsvlakken - $O(nlog(n))$ in plaats van $O(n)$ - dat bekeken wordt tijdens het bouwen.
De $\symBSParbitraryfastkd$ en $\symBSPclusterfastkd$ bomen tonen dat het kiezen van splitsingsrichtingen afhankelijk van de lokale geometrie deze boom nog beduidend beter kan maken.
Deze bomen tonen ook dat het toevoegen van slechts één richting bovenop de $\symKd$ richtingen, al een groot effect kan hebben op het aantal straal-driehoekintersecties en de rendertijd.
De clustering biedt geen voordeel ten opzichte van het kiezen van willekeurige normalen.

\section{Toekomstig onderzoek}
Toekomstig onderzoek zou zich moeten toespitsen op een aantal zaken: het verbeteren van de bouwtijd, het bepalen van het beste aantal richtingen en het bepalen van betere richtingen.
Het verbeteren van de bouwtijd zou kunnen door gebruik te maken van de methode van \authorBudge{} \cite{Budge} om de $\symSA$ van de gesplitste volumes incrementeel te berekenen in plaats van steeds helemaal vanaf nul.
Bepaalde delen van het bouwproces zijn ook goed paralleliseerbaar. 
Zo kan er in parallel over de verschillende splitsingsrichtingen gesweept worden.
Het aantal richtingen dat gegenereerd wordt zou afhankelijk gemaakt kunnen worden van de diepte.
Op hogere niveaus willen we voornamelijk $\symKd$ knopen dus is het mogelijk om daar enkel de $\symKd$ richtingen of de $\symKd$ richtingen plus een klein aantal $\symBSP$ richtingen te gebruiken. Op de lagere niveaus zouden dan meer $\symBSP$ richtingen gebruikt worden.\\

Het is ook een mogelijkheid om te stoppen met zoeken naar het beste splitsingsvlak, als er al een splitsingsvlak gevonden is dat de driehoeken opdeelt in twee disjuncte delen.
Dit laat toe om hogere $k$-waarden te kiezen, zonder een stijging in bouwtijd te krijgen.
In sectie \ref{h5-richtingen-kwaliteit} viel het ook op dat het aantal straal-driehoekintersecties sterk daalde als er één richting bepaald door de normalen gebruikt werd, maar dat meer richtingen aan de hand van de normalen, dit aantal niet meer lieten dalen.
Bij de $\symBSPrandomfastkd$ boom bleef het aantal intersecties dalen met een stijgend aantal richtingen.
Het lijkt ons dus een goed idee om een hybride oplossing te maken die een klein aantal richtingen bepaald aan de hand van de normalen en een groter aantal richtingen random genereert als er nog geen goede splitsing gevonden is.
%%% Local Variables: 
%%% mode: latex
%%% TeX-master: "masterproef"
%%% End: 