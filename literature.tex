\chapter{Literatuurstudie}
\label{hoofdstuk:literature}

\section{Raytracing}
Raytracing is een computergraphics techniek om tweedimensionale beelden te genereren van een virtuele driedimensionale wereld bestaande uit driehoeken.
Door elke pixel worden één (of meerdere) stralen gestuurd vanuit een oogpunt.
Voor elke straal wordt dan de dichtstbijzijnde intersectie met een driehoek in de scene berekend.
De kleur van de pixel is dan het gemiddelde van de kleuren van de intersectie punten van de stralen en driehoeken.

Om deze intersecties efficiënt te kunnen doen, wordt gebruik gemaakt van ruimtelijke en/of hiërarchische gegevensstructuren.
De drie hoofdklassen van acceleratie datastructuren in moderne raytracing zijn: Kd-bomen, uniform grid (mogelijks met meerdere niveaus) en as-gealigneerde bounding volume hiërarchieën (BVH). 
Verder wordt het raytracen versneld door: het sturen van pakketten coherente stralen, SIMD extensie van moderne CPU's gebruiken, Bounding Frustrum en interval aritmetica.
Het is nog onduidelijk hoe goed dit voor secundaire stralen werkt ???

Void area: Het verschil tussen de geprojecteerde oppervlakte van het bounding volume en de geprojecteerde oppervlakte van het omhulde object.
Het is aangetoond dat het aantal intersectie testen afhankelijk is van de void oppervlakte. [10]

\section{Binary Space Partitioning bomen}
Binary Space Partitioning bomen of afgekort BSP-bomen splitsen de ruimte op door willekeurig georiënteerde vlakken.
De bounding box van de scene wordt recursief onderverdeeld volgens willekeurige vlakken totdat een bepaalde stopconditie bereikt wordt.
Als enkel as-gealigneerde vlakken gebruikt worden, spreekt men van Kd-bomen.

BSP-bomen zijn in theorie op alle vlakke superieur ten opzicht van Kd-bomen:
\begin{itemize}
	\item De flexibiliteit om splitsingsvlakken te kunnen plaatsen waar ze het meest effectief zijn, zorgt ervoor dat BSP-bomen zich heel goed kunnen aanpassen aan complexe scenes en heel ongelijke scene distributies.
	\item Elke Kd-boom kan als een BSP-boom worden uitgedrukt.
\end{itemize}

Ondanks de theoretische superioriteit worden in de praktijk de Kd-bomen boven de BSP-bomen verkozen.
De extra flexibiliteit van de BSP-boom zorgt ervoor dat er $\symO(n^3)$ mogelijke splitsingsvlakken gecontroleerd moeten worden in tegenstelling tot $\symO(n)$ bij de Kd-boom.

\subsection{Een item}

\section{Besluit van dit hoofdstuk}
