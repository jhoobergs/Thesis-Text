\chapter{Voorgaand werk}
\label{hoofdstuk:voorgaand-werk}
Acceleratiestructuren zijn onmisbaar bij raytracing.
Dit hoofdstuk start met een algemene uitleg over acceleratiestructuren.
% TODO

\section{Acceleratiestructuren}
    Het doel van acceleratiestructuren is om het aantal straal-driehoek intersecties te verminderen.
    De simpelste acceleratiestructuur bestaat uit het omhullende volume van de scene.
    Intersecteren met de driehoeken uit de scene gebeurt dan enkel als dit omhullende volume intersecteert met de straal.
    Dit kan worden uitgebreid tot een boomstructuur door dit volume recursief op te delen in kindvolumes.
    Binaire bomen delen elk omhullend volume op in twee nieuwe volumes, andere acceleratiestructuren zoals bijvoorbeeld octrees delen het volume op in meer dan twee volumes.
    
    Het volume kan worden opgedeeld op twee manieren: volgens objecten of volgens ruimte.
    Bij opdeling volgens objecten worden de objecten binnen het volume opgedeeld in meerdere disjuncte groepen en de kindvolumes zijn de omhullende volumes van deze groepen.
    Na deze opdeling zit elk object in exact één van deze nieuwe volumes, maar de volumes kunnen overlappen.
    Een voorbeeld van een acceleratiestructuur waarbij de opdeling volgens objecten gebeurt is de \textit{Bounding Volume Hierarchy} ($\symBVH$).
    Opdeling volgens ruimte betekent dat de ruimte in het volume wordt opgedeeld in meerdere delen.
    Na deze opdeling overlappen deze nieuwe volumes niet, maar een object ligt nu in minstens één (en mogelijks in meerdere) kindvolume(s).
    De Binary Space Partioning ($\symBSP$) boom deelt de ruimte van het volume steeds op in twee kindvolumes.

    Een eenvoudige acceleratiestructuur bestaat uit een asgealigneerde balk die de scene omhult. 
    Als de straal niet intersecteert met de balk, dan raakt de straal geen enkele driehoek van de scene.
    Als de straal wel intersecteert met de balk, worden alle driehoeken in de omhullende balk getest voor intersectie.
    Deze balk kan recursief opgedeeld worden in meerdere balken om het aantal straal driehoek intersecties nog meer te verminderen.
    Als de straal intersecteert met de balk worden de kindbalken gecontroleerd in volgorde van hun afstand.
    Dit opdelingsproces heeft een aantal parameters: het aantal balken waarin een balk wordt opgedeeld, hoe de balk wordt opgedeeld en de stopconditie. \\

    Binaire bomen delen elke balk op in twee nieuwe balken, andere acceleratiestructuren zoals bijvoorbeeld octrees delen de balk op in meer dan twee balken.
    De balk kan opgedeeld worden op twee manieren: volgens object of volgens ruimte.
    Bij opdeling volgens objecten worden de objecten binnen de balk opgedeeld in meerdere groepen en de nieuwe balken zijn de omhullende balken van deze groepen.
    Na deze opdeling zit elk object in exact één van deze nieuwe balken, maar de balken kunnen overlappen.
    Deze acceleratiestructuur wordt \textit{Bounding Volume Hierarchy} ($\symBVH$) genoemd.
    Opdeling volgens ruimte betekent dat de ruimte van de balk wordt opgedeeld in meerdere delen.
    Na deze opdeling overlappen deze nieuwe balken niet, maar een object ligt nu in minstens één (en mogelijke meerdere) van de nieuwe balken.
    Deze acceleratiestructuur wordt de ${\symKd}$ boom genoemd.

    % TODO: combinatie van Kd en BVH ? SBVH
    % TODO: check if all abbrev are listed and defined at first usage
\section{$\symKd$ Bomen}    
    De $\symKd$ boom is een $\symBSP$ boom waarbij alle splitsingsvlakken asgealigneerd zijn.
    Hiermee wordt bedoelt dat de splitsingsrichting (de normaal op het splitsingsvlak) evenwijdig is aan één van de drie hoofdassen.
    Dit zorgt ervoor dat elke knoop van de boom een asgealigneerde balk voorstelt.
    Het doorkruisen van een knoop uit een $\symKd$ boom is daardoor goedkoper dan het doorkruisen van een knoop uit een algemene $\symBSP$ boom.
    De reden hiervoor is dat het goedkoper is om het intersectiepunt van een straal en een asgealigneerd vlak te vinden dan het intersectiepunt van een straal en een willekeurig vlak.

    Het opbouwen van een $\symKd$ boom bestaat uit het recursief opdelen van het omhullend volume.
    In elke stap moet bepaald worden volgens welke splitsingsrichting en op welke positie er gesplitst wordt.
    Het doel is om de totale intersectiekost van de boom te minimaliseren. 
    De heuristiek die dit doel probeert te bereiken is de Surface Area (SA) heuristiek.
    De SA heuristiek is oorspronkelijk ontwikkelt door REF voor de $\symBVH$ en later aangepast door REF voor de $\symKd$ boom.



    Meest onderzoek naar deze en gebruikt
    Beperkte versie: as-gealigneerd -> balken
    Voordelen: simpele intersectie splitsingsvlak
    SAH (inclusief nulbonus) -> sweeping
\section{RBSP-bomen}
    Algemener dan Kd -> meer richtingen -> richtingen gekozen volgens
    kDOPs ipv balken
    Praktische problemen: worst of both worlds -> trage doorkruising, 
    houd geen rekening met geometrie, kan nog steeds niet goed splitsen
\section{Algemeen principe}
    Spatiaal, opsplitsen volgens willekeurige vlakken
    Niet feasible geacht
\section{Algemene BSP-Bomen in de praktijk}
    Paper Ize et al.
        3 Kd richtingen (sweepen) plus 4 richtingen afhankelijk van geometrie 
        Snelle Kd traversal, aanpassing SAH
        Geen sweeping -> BVH als hulpstructuur
    
    Voordelen
        Knopen kunnen beter worden opgesplist
    Nadelen
        Duurder om te traversen
        Duurder om te bouwen (berekening SA van veelvlak vs SA van balk)
        Grote zoekruimte om beste splitsingsvlak te vinden
\section{Hiërarchie}


%%% Local Variables: 
%%% mode: latex
%%% TeX-master: "masterproef"
%%% End: 
