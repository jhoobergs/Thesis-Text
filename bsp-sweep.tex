\chapter{$\symBSPsweep$}
\label{hoofdstuk:bsp-sweep}
In dit hoofdstuk wordt een nieuwe soort $\symBSP$ boom besproken: de $\symBSPsweep$ boom.
Eerst wordt het nut van het opsplitsen van bladknopen in kleinere bladknopen wiskundig besproken.
Het algemene idee van de $\symBSPsweep$ boom wordt dan besproken en daarna worden een aantal specifieke versies van de $\symBSPsweep$ ($\symBSPrandom$, $\symBSParbitrary$ en $\symBSPcluster$) besproken.

% Duidelijk wiskundig uitleggen

 % TODO
 % Beginnen met het probleem te schetsen, aantal paragrafen aan spenderen: 
%  	- Eerst uitleggen in welke gevallen het nuttig is dat je die vrijheid gebruikt
%	- In dit soort scene, intuitief ...

\section{Probleemstelling}
    Het doel van acceleratiestructuren is om de totale tijd nodig om een scene te renderen, te minimaliseren.
    Deze rendertijd $\symTime_{\symRender}$ bestaat uit twee grote factoren: de tijd gespendeerd aan het intersecteren met driehoeken (de intersectietijd) en de tijd gespendeerd aan het doorkruisen van de boom (de doorkruistijd).
    De totale intersectietijd $\symTime_{\symIntersection, \symTotal}$ is afhankelijk van het aantal driehoeken $\symNbPrimitives_\symLeaf$ in de geïntersecteerde bladknopen $\symLeaf$ en het aantal keer $\symTraversalL_\symLeaf$ dat elk van deze bladknopen doorkruist wordt.
    De totale doorkruistijd $\symTime_{\symTraversal, \symTotal}$ is afhankelijk van het aantal doorkruisingen $\symTraversalL_{\symInternal}$ van inwendige knopen.
    Formule \ref{eq:rendertijd} beschrijft dit wiskundig met $\symTime_{\symIntersection}$ de tijd nodig voor één straal-driehoek intersectie en $\symTime_{\symTraversal}$ de tijd nodig om één knoop te doorkruisen. 
\begin{equation}
    \label{eq:rendertijd}
   \symTime_{\symRender} \sim
    \symTime_{\symIntersection, \symTotal} + \symTime_{\symTraversal, \symTotal} = \symTime_{\symIntersection} * \sum_{\symLeaf}^\symLeafL \symNbPrimitives_\symLeaf * \symTraversalL_\symLeaf + \symTime_{\symTraversal} * \symTraversalL_{\symInternal}
\end{equation}

Stel dat één kindknoop $\symLeaf_j$ uit de boom wordt opgesplitst in twee kleinere kindknopen $\symLeaf_{j1}$ en $\symLeaf_{j2}$ die elk de helft van de driehoeken krijgen.
Deze opsplitsing is voordelig als aan voorwaarde \ref{eq:vw_splitwinst} voldaan is.
Deze voorwaarde drukt uit dat knoop $\symLeaf_j$ nu een inwendig knoop wordt en dus niet meer zorgt voor een intersectietijd en wel voor een doorkruistijd en dat de nieuwe kindknopen zorgen voor een intersectietijd.
De voorwaarde is equivalent aan voorwaarden \ref{eq:vw_splitwinst_tussen} en \ref{eq:vw_splitwinst_res}.
% B -> 2Bs : nb -> nb/2 * 2 en Db1 + db2 << Db + 1 inwendige knoop, Db keer bekeken
\begin{equation}
    \label{eq:vw_splitwinst}
    \symTime_{\symRender} - \symTime_{\symIntersection} * \symNbPrimitives_{\symLeaf_j} * \symTraversalL_{\symLeaf_j} + \symTime_{\symTraversal} * \symTraversalL_{\symLeaf_j}  + \frac{\symNbPrimitives_{\symLeaf_j}}{2} * (\symTraversalL_{\symLeaf_{j1}} + \symTraversalL_{\symLeaf_{j2}}) * \symTime_{\symIntersection} \leq \symTime_{\symRender}
\end{equation}
\begin{equation}
    \label{eq:vw_splitwinst_tussen}
  \Leftrightarrow \frac{\symNbPrimitives_{\symLeaf_j}}{2} * (\symTraversalL_{\symLeaf_{j1}} + \symTraversalL_{\symLeaf_{j2}}) * \symTime_{\symIntersection} \leq (\symTime_{\symIntersection} * \symNbPrimitives_{\symLeaf_j} - \symTime_{\symTraversal}) * \symTraversalL_{\symLeaf_j}
\end{equation}
%\begin{equation}
%    \frac{\symNbPrimitives_{\symLeaf_j} * \symTime_{\symIntersection}}{(\symTime_{\symIntersection} * \symNbPrimitives_{\symLeaf_j} - \symTime_{\symTraversal})} \leq \frac{2\symTraversalL_{\symLeaf_j}}{(\symTraversalL_{\symLeaf_{j1}} + \symTraversalL_{\symLeaf_{j2}}) }
%\end{equation}
%\begin{equation}
%    \frac{1}{(1 - \frac{\symTime_{\symTraversal}}{\symNbPrimitives_{\symLeaf_j} * \symTime_{\symIntersection}})} \leq \frac{2\symTraversalL_{\symLeaf_j}}{(\symTraversalL_{\symLeaf_{j1}} + \symTraversalL_{\symLeaf_{j2}}) }
%\end{equation}
\begin{equation}
    \label{eq:vw_splitwinst_res}
    \Leftrightarrow \symTraversalL_{\symLeaf_{j1}} + \symTraversalL_{\symLeaf_{j2}} \leq
    2\symTraversalL_{\symLeaf_j} * (1 - \frac{\symTime_{\symTraversal}}{\symNbPrimitives_{\symLeaf_j} * \symTime_{\symIntersection}})
\end{equation}

De som in het linkerlid van voorwaarde \ref{eq:vw_splitwinst_res} is minstens gelijk aan $\symTraversalL_{\symLeaf_{j}}$ aangezien elke doorkruising van $\symLeaf_j$ voor minstens één doorkruising door een kindknoop zorgt.
Analoog kan worden ingezien dat de maximale waarde voor deze som gelijk is aan $2\symTraversalL_{\symLeaf_{j}}$ aangezien elke doorkruising van $\symLeaf_j$ voor maximaal twee doorkruisingen door een kindknoop kan zorgen. Hieruit volgt ongelijkheid \ref{eq:vw_doorkruisingen}.

\begin{equation}
    \label{eq:vw_doorkruisingen}
    \symTraversalL_{\symLeaf_{j}} \leq \symTraversalL_{\symLeaf_{j1}} + \symTraversalL_{\symLeaf_{j2}} \leq 2\symTraversalL_{\symLeaf_{j}}
\end{equation}

In het algemeen geval geldt dat $\symTraversalL_{\symLeaf_{j1}} + \symTraversalL_{\symLeaf_{j2}} = (2 * (1-\alpha) + \alpha) \symTraversalL_{\symLeaf_{j}}$ waarbij $\alpha$ het procentueel aantal doorkruisingen is dat door slechts één van de twee kindknopen gaat.
In dit geval leidt voorwaarde \ref{eq:vw_splitwinst_res} tot voorwaarde \ref{eq:vw_splitwinst_general}.
\begin{equation}
    \label{eq:vw_splitwinst_general}
    (2 * (1-\alpha) + \alpha) \symTraversalL_{\symLeaf_{j}} \leq
    2\symTraversalL_{\symLeaf_j} - \frac{2\symTraversalL_{\symLeaf_j}\symTime_{\symTraversal}}{\symNbPrimitives_{\symLeaf_j} \symTime_{\symIntersection}}
    \Leftrightarrow
    2 - \alpha \leq 2 - \frac{2\symTime_{\symTraversal}}{\symNbPrimitives_{\symLeaf_j} \symTime_{\symIntersection}}
    \Leftrightarrow
    \symTime_{\symTraversal} \leq \frac{\alpha \symNbPrimitives_{\symLeaf_j} }{2}\symTime_{\symIntersection}
\end{equation}

In het ideale geval ($\alpha = 1$) leidt voorwaarde \ref{eq:vw_splitwinst_general} tot $\symTime_{\symTraversal} \leq \frac{\symNbPrimitives_{\symLeaf_j}}{2}\symTime_{\symIntersection}$.
Het opsplitsen van een knoop met twee elementen, kan hierdoor pas voordelig zijn als de doorkruistijd kleiner is dan de intersectietijd.
Aangezien de doorkruistijd in de realiteit beduidend kleiner is dan de intersectietijd, is het in dit ideale geval altijd voordelig om een kindknoop op te splitsen, ongeacht het aantal driehoeken in de knoop.
In het slechtste geval ($\alpha = 0$) kan aan voorwaarde \ref{eq:vw_splitwinst_general} enkel voldaan zijn als de doorkruistijd gelijk is aan nul. Dit is onmogelijk waardoor opsplitsen nooit voordelig kan zijn in dit geval. \\

Het is moeilijk om de waarde van $\alpha$ te voorspellen, deze is namelijk afhankelijk van de exacte stralen die tijdens het renderen gevolgd worden, de specifieke driehoeken in de knoop en het splitsingsvlak. 
Voorwaarde \ref{eq:vw_splitwinst_general} toont dat de kans dat splitsen voordelig is, lineair stijgt met $\symNbPrimitives_{\symLeaf_j}$. 
De voorwaarde toont ook dat het splitsen van een knoop met twee driehoeken, voordelig is wanneer de doorkruistijd $\alpha$ keer kleiner is dan de intersectietijd. 
Intuitief lijkt het logisch dat hier in het algemeen aan voldaan is. 
Hieruit kan worden afgeleid dat het altijd beter is om bladknopen met meer dan één driehoek op te splitsen in twee kleinere bladknopen.\\

De volgende secties bespreken nieuwe $\symBSP$ bomen die gebruik maken van de vrijheid om op elk niveau van de boom andere splitsingsvlakken te gebruiken en op die manier meer bladknopen kunnen opsplitsen in kleinere bladknopen.\\



\section{Algemeen idee}
    De $\symBSPsweep$ boom is een algemene $\symBSP$ boom waarbij in elke knoop k richtingen bepaald worden en alle $2n$ splitsingsvlakken langs elk van deze richtingen worden bekeken door te sweepen.
    Deze k richtingen kunnen verschillend zijn voor elke knoop en kunnen gekozen worden afhankelijk van de lokale geometrie.
    De $\symRBSP$ boom is een $\symBSPsweep$ boom waarbij de gekozen richtingen in elke knoop hetzelfde zijn.
    De $\symBSPsweep$ boom heeft drie belangrijke ontwerpbeslissingen.
    De belangrijkste ontwerpbeslissing bij de $\symBSPsweep$ boom is de methode die gebruikt wordt om de k richtingen te bepalen.
    Een tweede belangrijke ontwerpbeslissing is de waarde van k.
    De derde belangrijke ontwerpbeslissing sluit aan bij de eerst en gaat over het al dan niet gebruiken van de $\symKd$ richtingen als de eerste drie van de k richtingen.
    \\


    %Kd-richtingen + aantal richtingen of puur die richtingen
    %Snelle traversal voor kd-richtingen
\section{Gebaseerd op random richtingen}
De simpelste $\symBSPsweep$ boom, de $\symBSPrandom$ boom, bepaalt in elke knoop k random richtingen onafhankelijk van de geometrie.
De richtingen worden uniform op de hemisphere gegenereerd.
Het idee achter deze boom is dat het nuttiger kan zijn om driehoeken via veel verschillende vlakken te proberen splitsen, dan om ze steeds met dezelfde vlakken te proberen splitsen.
Als de driehoeken in de scene uniform verdeeld zijn, dan is de kans dat twee driehoeken volgens een willekeurige richting gesplitst kunnen worden, even groot als de kans dat ze door een $\symKd$ richting gesplitst kunnen worden.\\

Als de $\symBSPrandom$ boom perfect gebalanceerd is, worden in elk niveau $2kn$ verschillende splitsingsvlakken bekeken.
Deze splitsingsvlakken zijn verschillend op elk niveau, zodat in totaal $2knlog(n)$ verschillende splitsingsvlakken bekeken worden.
Figuur \ref{fig:splitsingsvlakken-bsprandom} toont dit visueel.
De $\symBSPrandom$ boom probeert elke driehoek via gemiddeld $2klog(n)$ ($\symO(log(n))$) vlakken te splitsen van de andere driehoeken, in tegenstelling tot de bestaande bomen die dit maximaal met $\symO(1)$ vlakken proberen.\\

\begin{figure}
    \centering

   \resizebox{0.4\textwidth}{!}{%
   \centering
    \begin{tikzpicture}
   \Tree
   [ .\node[fill=green]{$2kn$};    
       [.\node[fill=nodeblue1]{$2k\frac{n}{2}$};
           [.\node(n3l)[fill=nodered1, text=white]{$2k\frac{n}{4}$};]
           [.\node[fill=nodered2, text=white]{$2k\frac{n}{4}$};]
       ]
       [.\node[fill=nodeblue2, text=white]{$2k\frac{n}{2}$};
           [.\node[fill=nodered3, text=white]{$2k\frac{n}{4}$};]
           [.\node(n3r)[fill=nodered4, text=white]{$2k\frac{n}{4}$};]
       ]
   ]
   \node at (3,0.5) {$\#$};
   \node at (4,0.5) {$\# \neq$};
   \node at (3,0) {$2kn$};
   \node at (3,-1.5) {$2kn$};
   \node at (3,-3) {$2kn$};
   \node at (4,0) {$2kn$};
   \node at (4,-1.5) {$4kn$};
   \node at (4,-3) {$6kn$};
   \end{tikzpicture}
   }
   \caption[Splitsingsvlakken $\symBSPrandom$]%
    {Splitsingsvlakken $\symBSPrandom$ - \small Per niveau het aantal ($\#$) splitsingsvlakken en het totaal aantal verschillende ($\# \neq$) splitsingsvlakken gebruikt in bovenliggende niveaus bij de $\symBSPrandom$ boom.} %TODO: meer uitleg
    \label{fig:splitsingsvlakken-bsprandom}
\end{figure}

\begin{figure}
        \resizebox{\textwidth}{!}{%
        \centering
         \begin{tikzpicture}
            \usetikzlibrary{arrows}
            \usetikzlibrary{shapes}
\tikzstyle{every tree node}=[draw, ellipse, align=center]
        \Tree
        [ .\node[shading = axis, left color=green, right color=nodeyellow1,shading angle=135]{$2(k-3)n + 6n$};    
            [.\node[shading = axis, left color=nodeblue1, right color=nodeyellow1, shading angle=135]{$2(k-3)\frac{n}{2} + 6\frac{n}{2}$};
                [.\node(n3l)[shading = axis, left color=nodered1, right color=nodeyellow1, shading angle=135]{$2(k-3)\frac{n}{4} + 6\frac{n}{4}$};]
                [.\node[shading = axis, left color=nodered2, right color=nodeyellow1, shading angle=135]{$2(k-3)\frac{n}{4} + 6\frac{n}{4}$};]
            ]
            [.\node[shading = axis, left color=nodeblue2, right color=nodeyellow1, shading angle=135, text=white]{$2(k-3)\frac{n}{2} + 6\frac{n}{2}$};
                [.\node[shading = axis, left color=nodered3, right color=nodeyellow1, shading angle=135]{$2(k-3)\frac{n}{4} + 6\frac{n}{4}$};]
                [.\node(n3r)[shading = axis, left color=nodered4, right color=nodeyellow1, shading angle=135]{$2(k-3)\frac{n}{4} + 6\frac{n}{4}$};]
            ]
        ]
        \node at (10,0.5) {$\#$};
        \node at (13,0.5) {$\# \neq$};
        \node at (10,0) {$2(k-3)n + 6n$};
        \node at (10,-1.5) {$2(k-3)n + 6n$};
        \node at (10,-3) {$2(k-3)n + 6n$};
        \node at (13,0) {$2(k-3)n + 6n$};
        \node at (13,-1.5) {$4(k-3)n + 6n$};
        \node at (13,-3) {$6(k-3)n + 6n$};
        \end{tikzpicture}
        }
    \caption[Splitsingsvlakken $\symBSPrandomsomekd$]%
    {Splitsingsvlakken $\symBSPrandomsomekd$ - \small Per niveau het aantal ($\#$) splitsingsvlakken en het totaal aantal verschillende ($\# \neq$) splitsingsvlakken gebruikt in bovenliggende niveaus bij de $\symBSPrandomsomekd$ boom.} %TODO: meer uitleg
    \label{fig:splitsingsvlakken-bsprandomsomekd}
\end{figure}

De $\symKd$ richtingen zijn in praktische scenes vaak beter dan willekeurige richtingen omdat ze loodrecht op elkaar staan waardoor ze de hemisphere goed bedekken en omdat scenes die door de mens gemaakt worden, vaak asgealigneerde delen bevatten.
Dit geeft aanleiding tot een boom die als eerste drie richtingen steeds de $\symKd$ richtingen kiest en enkel de overige $k - 3$ richtingen random genereert: de $\symBSPrandomkd$ boom. Een extra voordeel is dat de snellere $\symKd$ knoop doorkruising gebruikt kan worden. Een $\symBSPrandom$ boom die altijd de $\symKd$ richtingen gebruikt en de doorkruising van $\symKd$ knopen optimaliseert, wordt aangeduid als $\symBSPrandomfastkd$. 
Als de $\symBSPrandomsomekd$ boom perfect gebalanceerd is, worden in elk niveau $2kn$ splitsingsvlakken bekeken.
Van deze $2kn$ zijn er $6n$ die hergebruikt worden, de vlakken volgens de $\symKd$ richtingen.
Dit zorgt voor $2(k-3)n$ verschillende splitsingsvlakken per niveau, zodat in totaal $2(k-3)nlog(n) + 6n$ verschillende splitsingsvlakken bekeken worden.
Figuur \ref{fig:splitsingsvlakken-bsprandomsomekd} toont dit visueel.
De $\symBSPrandomsomekd$ boom probeert elke driehoek via gemiddeld $\symO(log(n))$ vlakken te splitsen van de andere driehoeken, net als  de $\symBSPrandom$ boom.\\


%(Extra) richtingen door random richtingen te kiezen
%Ter controle dat de richtingen met behulp van normalen, nuttige richtingen zijn
%Sweeping
%Waarom zou dit werken ? : Random per node itt vast bij Kd
%    Driehoeken proberen te worden gesplitst volgens meer verschillende richtingen, Kd probeert steeds hetzelfde
%    Kd heeft maar 3 opties, als het volgens geen kan -> nooit mogelijk
%    Kans splitsbaar door Kd richting of random is even hoog in uniform geval. Scenes hebben wel veel asgealigneerde delen, dus daarom die extra.

\section{Gebaseerd op normalen}
    De kracht van de $\symBSPsweep$ boom ligt in het feit dat er gebruik gemaakt kan worden van de lokale geometrie om de splitsingsrichtingen te bepalen.
    De normalen van de driehoeken in een knoop bevatten informatie over de oriëntatie van de driehoeken.
    De autopartitie vlakken van \authorIze{} maken ook gebruik van de normalen, maar de normaal van een driehoek wordt enkel voor die driehoek zelf gebruikt.
    
\subsection{Willekeurige normaal}
    De $\symBSParbitrary$ boom is een $\symBSPsweep$ boom waarbij in elke knoop de normalen van k willekeurige driehoeken gekozen worden als splitsingrichtingen.
    Als er k of minder driehoeken in de knoop zitten, worden alle normalen als splitsingrichting genomen.
    In het algemeen zijn er n driehoeken met n verschillende normalen waardoor er in totaal $2n^2$ mogelijke splitsingsvlakken zijn.
    Figuur \ref{fig:splitsingsvlakken-bsparbitrary-top} toont het aantal splitsingsvlakken gebruikt per knoop in de bovenste niveaus van de boom.
    Deze figuur is identiek aan de figuur voor de $\symBSPrandom$ boom. \\

    Voor de onderste $log(k)$ niveaus is er echter een verschil, omdat er dan k of minder driehoeken in elke knoop zitten.
    In de onderste $log(k)$ niveaus worden er in elke knoop $2n_m^2$ splitsingsvlakken bekeken, met $n_m <= k$.
    Een knoop op het $m^{de}$ laagste niveau (met $m \leq log(k)$) bevat $\frac{n}{2^{log(n) - m}}$ driehoeken waardoor er $2n_m^2 = 2 * (\frac{n}{2^{log(n) - m}})^2 = 2*(2^m)^2 = 2*4^m$ splitsingen gebeuren in deze knoop.
    Formule \ref{eq:bsparbitrary-logkjonderste} toont dat het totaal aantal splitsingsvlakken bekeken op het ${log(k) - j}^{de}$ onderste niveau gelijk is aan $\frac{2kn}{2^j}$. Dit aantal is berekend als het aantal splitsingsvlakken per knoop ($2*4^{log(k) - j}$) vermenigvuldigd met het aantal knopen ($2^{log(n)-log(k)+j}$) op het niveau.
    \begin{equation}
        \label{eq:bsparbitrary-logkjonderste}
    2*4^{log(k) - j} * 2^{log(n)-log(k)+j} = 2 * 2^{log(k) + log(n) - j} = \frac{2 * 2^{log(kn)}}{2^j} =\frac{2kn}{2^j}
    \end{equation}
    In de bovenste $log(n) - log(k) = log(\frac{n}{k})$ niveaus worden $2log(\frac{n}{k})kn$ verschillende splitsingsvlakken bekeken.
    Het niveau eronder gebruikt nog $kn$ nieuwe splitsingsvlakken zodat in totaal $2(log(\frac{n}{k}) + \frac{1}{2})kn$ verschillende splitsingsvlakken worden gebruikt.
    Figuur \ref{fig:splitsingsvlakken-bsparbitrary-bottom} toont dit visueel.
    Als de boom perfect gebalanceerd is, probeert de $\symBSParbitrary$ boom elke driehoek via gemiddeld $2klog(\frac{n}{k} + \frac{1}{2})$ ($\symO(log(n))$) vlakken te splitsen van de andere driehoeken. Dit aantal is minder dan bij de $\symBSPrandom$ boom, die ook in de lagere niveaus steeds k splitsingrichtingen genereert.\\

    Net als bij de $\symBSPrandom$ boom kan een versie van de $\symBSParbitrary$ boom gemaakt worden die als eerste drie richtingen steeds de $\symKd$ richtingen kiest en enkel voor de overige $k - 3$ richtingen willekeurige normalen selecteert: de $\symBSParbitrarykd$ boom. Een $\symBSParbitrary$ boom die altijd de $\symKd$ richtingen gebruikt en de doorkruising van $\symKd$ knopen optimaliseert, wordt aangeduid als $\symBSParbitraryfastkd$. Als de $\symBSParbitrarysomekd$ boom perfect gebalanceerd is, worden in totaal $2(log(\frac{n}{k-3}) + \frac{1}{2})(k-3)n + 6n$ verschillende splitsingsvlakken bekeken. De analyse hiervoor is analoog aan de analyse voor $\symBSPrandomsomekd$.

    %TODO: subfig a en b, a zelfde als bsprandom, b = laatste log(k) niveaus -> allemaal dezelfde
    %TODO: kd verbetering -> zelfde formule maar k-3 en + 6n


\begin{figure}
    \centering
    \begin{subfigure}[t]{0.34\textwidth}

   \resizebox{\textwidth}{!}{%
   \centering
    \begin{tikzpicture}
   \Tree
   [ .\node[fill=green]{$2kn$};    
       [.\node[fill=nodeblue1]{$2k\frac{n}{2}$};
           [.\node(n3l)[fill=nodered1, text=white]{$2k\frac{n}{4}$};]
           [.\node[fill=nodered2, text=white]{$2k\frac{n}{4}$};]
       ]
       [.\node[fill=nodeblue2, text=white]{$2k\frac{n}{2}$};
           [.\node[fill=nodered3, text=white]{$2k\frac{n}{4}$};]
           [.\node(n3r)[fill=nodered4, text=white]{$2k\frac{n}{4}$};]
       ]
   ]
   \node at (3,0.5) {$\#$};
   \node at (4,0.5) {$\# \neq$};
   \node at (3,0) {$2kn$};
   \node at (3,-1.5) {$2kn$};
   \node at (3,-3) {$2kn$};
   \node at (4,0) {$2kn$};
   \node at (4,-1.5) {$4kn$};
   \node at (4,-3) {$6kn$};
   \end{tikzpicture}
   }
   \caption{Bovenste drie niveaus}
   \label{fig:splitsingsvlakken-bsparbitrary-top}
\end{subfigure}
\begin{subfigure}[t]{0.64\textwidth}

    \resizebox{\textwidth}{!}{%
    \centering
     \begin{tikzpicture}
        \usetikzlibrary{arrows}
        \usetikzlibrary{shapes}
\tikzstyle{every tree node}=[draw, ellipse, align=center]
    \Tree
    [ .\node[shading = axis, left color=nodeblue1, right color=nodered2,shading angle=90, text=white]{$2*4^{log(k)}$};    
        [.\node[shading = axis, left color=nodeblue1, right color=nodeblue2,shading angle=90, text=white]{$2*4^{log(k) - 1}$};
            [.\node(n3l)[fill=nodeblue1, text=black]{$2*4^{1}$};]
            [.\node[fill=nodeblue2, text=white]{$2*4^{1}$};]
        ]
        [.\node[shading = axis, left color=nodered1, right color=nodered2,shading angle=90, text=white]{$2*4^{log(k) - 1}$};
            [.\node[fill=nodered1, text=white]{$2*4^{1}$};]
            [.\node(n3r)[fill=nodered2, text=white]{$2*4^{1}$};]
        ]
    ]
    \node at (5,0.5) {$\#$};
    \node at (8,0.5) {$\# \neq$};
    \node at (5,0) {$2kn$};
    \node at (5,-1.5) {$kn$};
    \node at (5,-3) {$\frac{kn}{2}$};
    \node at (8,0) {$2(log(\frac{n}{k}) + \frac{1}{2})kn$};
    \node at (8,-1.5) {$2(log(\frac{n}{k}) + \frac{1}{2})kn$};
    \node at (8,-3) {$2(log(\frac{n}{k}) + \frac{1}{2})kn$};
    \end{tikzpicture}
    }
    \caption{Onderste $log(k)$ niveaus}
    \label{fig:splitsingsvlakken-bsparbitrary-bottom}
 \end{subfigure}
   \caption[Splitsingsvlakken $\symBSParbitrary$]%
    {Splitsingsvlakken $\symBSParbitrary$ - \small Per niveau het aantal ($\#$) splitsingsvlakken en het totaal aantal verschillende ($\# \neq$) splitsingsvlakken gebruikt in bovenliggende niveaus bij de $\symBSPrandom$ boom.} %TODO: meer uitleg
\end{figure}

    

    %Extra richtingen door random normalen te kiezen
    %Autopartitie van Ize maar gesweeped
    %Waarom zou dit werken ? : ...
    
\subsection{Geclusterde normalen}
    Bovenstaande $\symBSPsweep$ bomen bevatten een grote niet-deterministische factor.
    De $\symBSPcluster$ boom gebruikt het K-means clustering algoritme om deterministischere richtingen te bepalen.
    De normalen van de driehoeken in een knoop worden geclusterd in k clusters en de centra van deze clusters worden gebruikt als richtingen voor de $\symBSPsweep$ boom.
    Als er k of minder driehoeken in de knoop zitten, worden alle normalen als splitsingsrichting genomen, net zoals bij de $\symBSParbitrary$ boom.
    In tegenstelling tot de $\symBSParbitrary$ is het voor de $\symBSPcluster$ moeilijker om het totaal aantal mogelijke splitsingsvlakken te bepalen.
    In het algemeen kan elke clustering verschillende richtingen geven waardoor er oneindig veel mogelijke splitsingsrichtingen mogelijk zijn.
    De $\symBSPcluster$ boom is zeer gelijkaardig aan de $\symBSParbitrary$ boom waardoor het totaal aantal verschillende splitsingsvlakken die gebruikt worden, gelijk is. Analoog bestaan ook de $\symBSPclusterkd$ en $\symBSPclusterfastkd$ bomen.



    %TODO, identiek dezelfde afleidingen als arbitrary
    %Waarom zou dit werken ...

% TODO: vergelijking

\section{Vergelijking}
Tabel \ref{tab:boom-vergelijking-sweep} vergelijkt de hierboven besproken $\symBSPsweep$ bomen.
Alle drie bomen zijn algemene $\symBSP$ bomen die een convex veelvlak hebben als omhullend volume.
De $\symBSPsweep$ bomen zijn ontworpen om sweeping te ondersteunen en zo efficiënt de $SAH$ kosten te kunnen berekenen, zonder hulpstructuren.
De $\symBSPrandom$ boom is onafhankelijk van de lokale geometrie, de twee andere bomen bepalen hun splitsingsvlakken afhankelijk van de lokale geometrie.
De $\symBSPsweep$ bomen zijn makkelijk uit te breiden naar $\symBSPsweepkd$ bomen die de optimalisatie met de snelle $\symKd$ knoop doorkruising gebruiken.
Per niveau gebruiken de $\symBSPsweep$ bomen niet altijd meer splitsingsvlakken dan de bestaande bomen, maar de $\symBSPsweep$ bomen gebruiken in totaal duidelijk meer verschillende splitsingsvlakken dan de bestaande bomen en zouden zich hierdoor beter moeten kunnen aanpassen aan complexe scenes.

\begin{table}[tb]
    \centering
    \begin{tabular}{@{}|l|c|c|c|@{}} \toprule      
            & $\symBSPrandom$     & $\symBSParbitrary$ & $\symBSPcluster$ \\ \midrule
      Omhullend volume & Convex veelvlak & Convex veelvlak & Convex veelvlak \\
      Sweeping                              &  Ja   & Ja & Ja    \\
      Geometrie afhankelijke vlakken & Nee & Ja & Ja \\
      Snelle Kd doorkruising                 & $\symBSPrandomfastkd$  & $\symBSParbitraryfastkd$  & $\symBSPclusterfastkd$    \\
      \# splitsingsvlakken per niveau       &  $2kn$   & $2kn$* & $2kn$*  \\
      totaal \# $\neq$ splitsingsvlakken           &  $2knlog(n)$   & $2(log(\frac{n}{k}) + \frac{1}{2})kn$ & $2(log(\frac{n}{k}) + \frac{1}{2})kn$     \\ \bottomrule
    \end{tabular}
    \caption[Vergelijking van de nieuwe soorten $\symBSP$ bomen.]{Vergelijking van de nieuwe soorten $\symBSP$ bomen - \small Deze tabel vat een aantal belangrijke eigenschappen van de nieuwe soorten $\symBSP$ bomen samen. *De onderste niveaus van $\symBSParbitrary$ en $\symBSPrandom$ bekijken minder dan $2kn$ splitsingsvlakken.}
    \label{tab:boom-vergelijking-sweep}
  \end{table}


  \begin{tikzpicture}[scale=0.6]
    \begin{axis}[
      axis lines = left,
      ylabel = \small Cumulatieve Procentuele bijdrage aan totaal aantal driehoekintersecties,
      xlabel = \small Aantal driehoeken in leafnode,
      ymin=0, ymax=100,
      width=0.8\textwidth,
      height=\textwidth,
      cycle multi list={%
        color list\nextlist
        [2 of]mark list
      },
      title = \small Invloed grootte bladknopen op totaal aantal intersecties,
      legend pos=south east,
  ]
    \addplot table [x=V1, y=KillerooKd, col sep=comma] {data/killeroo_feet_kd_bsp.csv};
    \addlegendentry{$\symKd$ Killeroo}
    \addplot table [x=V1, y=FeetKd, col sep=comma] {data/killeroo_feet_kd_bsp.csv};
    \addlegendentry{$\symKd$ Killeroo Been}

    \addplot table [x=V1, y=KillerooBSP, col sep=comma] {data/killeroo_feet_kd_bsp.csv};
    \addlegendentry{$\symBSPizefastkd$ Killeroo}
    \addplot table [x=V1, y=FeetBSP, col sep=comma] {data/killeroo_feet_kd_bsp.csv};
    \addlegendentry{$\symBSPizefastkd$ Killeroo Been}
    
      \end{axis}
    \end{tikzpicture}
    \begin{tikzpicture}[scale=0.6]
        \begin{axis}[
          axis lines = left,
          ylabel = \small Cumulatieve bijdrage aan totaal aantal driehoekintersecties,
          xlabel = \small Aantal driehoeken in leafnode,
          ymin=1000000, ymax=180000000,
          width=0.8\textwidth,
          height=\textwidth,
          cycle multi list={%
            color list\nextlist
            [2 of]mark list
          },
          title = \small Invloed grootte bladknopen op totaal aantal intersecties,
          legend pos=north west,
      ]
        \addplot table [x=V1, y=KillerooKdTotal, col sep=comma] {data/killeroo_feet_kd_bsp.csv};
        \addlegendentry{$\symKd$ Killeroo}
        \addplot table [x=V1, y=FeetKdTotal, col sep=comma] {data/killeroo_feet_kd_bsp.csv};
        \addlegendentry{$\symKd$ Killeroo Been}
    
        \addplot table [x=V1, y=KillerooBSPTotal, col sep=comma] {data/killeroo_feet_kd_bsp.csv};
        \addlegendentry{$\symBSPizefastkd$ Killeroo}
        \addplot table [x=V1, y=FeetBSPTotal, col sep=comma] {data/killeroo_feet_kd_bsp.csv};
        \addlegendentry{$\symBSPizefastkd$ Killeroo Been}
        
          \end{axis}
        \end{tikzpicture}
%%% Local Variables: 
%%% mode: latex
%%% TeX-master: "masterproef"
%%% End: 