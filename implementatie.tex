\chapter{Implementatie}
\label{hoofdstuk:implementatie}
In dit hoofdstuk wordt de implementatie van volgende $\symBSP$ bomen besproken: $\symKd$ boom, $\symRBSP$ boom, $\symRBSPKd$ boom, $\symBSPize$ boom, $\symBSPizefastkd$ boom, $\symBSPrandom$ boom, $\symBSPrandomkd$ boom, $\symBSPrandomfastkd$ boom, $\symBSParbitrary$ boom, $\symBSParbitrarykd$ boom, $\symBSParbitraryfastkd$ boom, $\symBSPcluster$ boom, $\symBSPclusterkd$ boom en $\symBSPclusterfastkd$ boom.
Deze implementaties zijn toegevoegd aan de pbrt-v3 \textit{ray tracer} \cite{pbrt} en zijn beschikbaar op \url{https://github.com/jhoobergs/Thesis-pbrt-v3}.\\

Het hoofdstuk start met een hoog niveau beschrijving van het algoritme om een $\symBSP$ boom te bouwen en het algoritme om deze bomen te intersecteren.
Daarna wordt voor elke type apart besproken hoe hun knopen voorgesteld worden in het geheugen, hoe inwendige knopen doorkruist worden en hoe het beste splitsingsvlak bepaald wordt.

\definecolor{codegreen}{rgb}{0,0.6,0}
\definecolor{codegray}{rgb}{0.5,0.5,0.5}
\definecolor{codepurple}{rgb}{0.58,0,0.82}
\definecolor{backcolor}{rgb}{0.95,0.95,0.92}
\lstdefinestyle{pseudoStyle}{
    backgroundcolor=\color{backcolor},   
    numberstyle=\tiny\color{codegray},
    stringstyle=\color{codepurple},
    %basicstyle=\footnotesize,
    breakatwhitespace=false,         
    breaklines=true,                 
    captionpos=b,                    
    keepspaces=true,                 
    numbers=left,                    
    numbersep=5pt,                  
    showspaces=false,                
    showstringspaces=false,
    showtabs=false,                  
    tabsize=2,
    basicstyle=\fontsize{9}{11}\selectfont\ttfamily
}
\lstset{style=pseudoStyle}

\section{Hoog niveau beschrijving}
\label{sec:h4-hoog-niveau}
\subsection{Bouwalgoritme}
Het bouwalgoritme voor $\symBSP$ bomen heeft een aantal parameters:
\begin{itemize}
    \item $\symCost_\symIntersection$: De intersectiekost voor primitieven. Deze parameter is nodig voor de $\symSAH$ heuristiek.
    \item $\symCostTraversalKd$: De doorkruiskost voor $\symKd$ knopen. Deze parameter is nodig voor de $\symSAH$ heuristiek.
    \item $\symCostTraversalBSP$: De aparte doorkruiskost voor $\symBSP$ knopen. Deze parameter is nodig voor de $\symSAH$ heuristiek.
    \item $\symMaxPrims$: Elke knoop met $\symMaxPrims$ of minder primitieven wordt direct een bladknoop, er wordt niet geprobeerd om de knoop te splitsen.
    \item $\symMaxDepth$: De maximale diepte van de boom.
    \item $maxIteraties$: Het maximale aantal iteraties bij de K-means clustering van de $\symBSPcluster$ boom.
    \item $\alpha$: Een parameter om $\symCostTraversalBSP$ lineair te laten variëren met het aantal driehoeken. Deze parameter wordt gebruikt bij het bouwen van bomen met de snelle $\symKd$ knoopdoorkruising.
\end{itemize}
\begin{dutchalgorithm}
    \begin{algorithmic}
        \State $stack\gets \emptyset$
        \State Voeg een bouwknoop met alle primitieven toe aan de stack
        \While {$stack \neq \emptyset$}
            \State $b \gets $\Call{pop}{$stack$}
            \If {$b_{\symNbPrimitives} \leq \symMaxPrims$ \Or $b_{d} = \symMaxDepth$}
                \State \Call{maak\_blad\_knoop}{b}
                \State $continue$
            \EndIf
            \State $besteSplit \gets $ \Call{bepaal\_beste\_splitsing}{b}
            \State $nietSplitKost \gets  b_{\symNbPrimitives}*\symCost_\symIntersection$
            \If {$besteSplit_{kost} > nietSplitKost$}
                \State $b_{slechteAanpassingen} \gets b_{slechteAanpassingen} + 1$
            \EndIf
            \If {$(besteSplit_{kost} > 4 * nietSplitKost$ \And $b_{\symNbPrimitives} < 16)$ \Or $besteSplit = None$ \Or $b_{slechteAanpassingen} = 3$}
                \State \Call{maak\_blad\_knoop}{b}
                \State $continue$
            \EndIf
            \State \Call{maak\_inwendige\_knoop}{b}
            \State Plaats de kindknopen als twee nieuwe bouwknopen op de stack
        \EndWhile
    \end{algorithmic}
    \caption{Bouwen van een BSP boom}
    \label{alg:bsp-bouw}
\end{dutchalgorithm}

Algoritme \ref{alg:bsp-bouw} toont de algemene vorm van het bouwalgoritme voor $\symBSP$ bomen.
Elk type $\symBSP$ boom wordt bepaald door de specifieke implementatie van de volgende functies:
\begin{enumerate}
    \item BEPAAL\_BESTE\_SPLITSING(bouwknoop): Deze functie bepaalt de beste splitsing voor de knoop. Het resultaat bevat het splitsingsvlak en de bijhorende $\symSAH$ kost.
    \item MAAK\_BLAD\_KNOOP(bouwknoop): Deze functie maakt een bladknoop van de huidige bouwknoop.
    \item MAAK\_INWENDIGE\_KNOOP(bouwknoop): Deze functie maakt een inwendige knoop van de huidige bouwknoop.
\end{enumerate}
De eerste functie omvat de specifieke eigenschappen van het type $\symBSP$ boom en bepaalt de kracht van dat type $\symBSP$ boom. 
Het resultaat van de tweede en derde functie is een specifieke representatie voor bladknopen / inwendige knopen die nuttig is voor het specifieke type $\symBSP$ boom.
In de volgende secties wordt bij elk type $\symBSP$ boom hun specifieke knooprepresentaties besproken.
De exacte implementaties van de functies om deze representaties in te vullen, zijn triviaal en worden niet expliciet beschreven.\\


Samengevat gaat Algoritme \ref{alg:bsp-bouw} voor elke knoop, die nog voldoende primitieven bevat en niet op de maximale diepte ligt, de beste splitsing bepalen en afhankelijk van de $\symSAH$ waarde bepalen of een splitsing moet gebeuren. 
Aangezien een slechte splitsing op een bepaald niveau, kan leiden tot een nuttige splitsing op een lager niveau, wordt een knoop toch gesplitst als splitsen nadelig is volgens de $\symSAH$ heuristiek.
Zulke \textit{slechte aanpassingen} worden niet gedaan als de kost van de beste splitsing vier keer hoger is dan de kost om niet te splitsen en als er bovendien minder dan 16 primitieven in de knoop zitten.
Elk lusvrij pad van de wortelknoop naar elke andere knoop, mag maximaal twee zulke \textit{slechte aanpassingen} bevatten.
Dit komt neer op het feit dat een \textit{slechte aanpassing} enkel mag gebeuren als er minder dan twee voorouderknopen een \textit{slechte aanpassing} gedaan hebben.
Dit concept is overgenomen van de originele implementatie van de $\symKd$ boom in \textit{pbrt-v3}.

\subsection{Intersectie-algoritme}
Het intersectie-algoritme bepaalt het intersectiepunt tussen een straal en de $\symBSP$ boom.
Een straal wordt voorgesteld met de parametervoorstelling $\vec{s} = \vec{o} + t\vec{d}$ met $\vec{o}$ het startpunt en $\vec{d}$ de richtingsvector.
Algoritme \ref{alg:bsp-intersectie} toont de algemene vorm van het intersectie-algoritme voor $\symBSP$ bomen.
Dit intersectie-algoritme is ontworpen voor zichtstralen en bepaalt dus het dichtste intersectiepunt.
De aanpassing naar een algoritme dat controleert of er een intersectiepunt bestaat, in plaats van te zoeken naar het dichtstbijzijnde, is triviaal en wordt niet verder besproken.
Deze aanpassing is nodig voor de intersectie van schaduwstralen.\\

\begin{dutchalgorithm}
    \begin{algorithmic}       
        \Function{intersecteer}{Boom b, Straal s}
            \State $geraakt_{omhullendVolume}, tMin, tMax \gets $ \Call{intersecteer}{$b_{omhullendVolume}$, s}
            \If {not $geraakt_{omhullendVolume}$}
                \State \Return false
            \EndIf
            \State $geraakt \gets false$
            \State $k \gets b_{wortelKnoop}$
            \State $stack\gets \emptyset$

            \While {$k \neq None$}
                \If {$s_{maxT} < tMin$}
                    \State $break$
                \EndIf
                \If {$k_{isInwendig}$}
                    \State $tVlak, linksEerst \gets$ \Call{doorkruis\_inwendige\_knoop}{k,s}
                    \If {$linksEerst$}
                        \State $k_1 \gets k_{linkerKind}$
                        \State $k_2 \gets k_{rechterKind}$
                    \Else
                        \State $k_1 \gets k_{rechterKind}$
                        \State $k_2 \gets k_{linkerKind}$
                    \EndIf
                    \If {$tVlak > tMax$ \Or $tVlak <= 0$}
                        \State $k \gets k_1$
                    \ElsIf {$tVlak < tMin$}
                        \State $k \gets k_2$
                    \Else
                        \State \Call{add}{$stack, \{k_2, tMin: tVlak, tMax: tMax\})$}
                        \State $k \gets k_1$
                        \State $tMax \gets tVlak$ 
                    \EndIf
                \Else
                    \If {\Call{intersecteer\_bladknoop}{k, s}}
                        \State $geraakt \gets true$
                    \EndIf
                    \If {$stack \neq \emptyset$}
                        \State $k, tMin, tMax \gets $\Call{pop}{$stack$}
                    \Else
                        \State $k \gets None$
                    \EndIf
                \EndIf
            \EndWhile
            \State \Return $geraakt$
        \EndFunction
    \end{algorithmic}
    \caption{Intersecteren van een BSP boom}
    \label{alg:bsp-intersectie}

\end{dutchalgorithm}  

Het algoritme maakt gebruik van de volgende functies:
\begin{enumerate}
    \item DOORKRUIS\_INWENDIGE\_KNOOP(knoop, straal): Deze functie bepaalt de intersectie van de gegeven straal met het splitsingsvlak van de gegeven inwendige knoop. Het resultaat bevat tVlak, de t-waarde waarvoor de straal intersecteert met het vlak en linksEerst, een booleaanse waarde die aangeeft of de straal eerst door de linkerkindknoop gaat of niet.
    \item INTERSECTEER\_BLAD\_KNOOP(knoop, straal): Deze functie bepaalt de intersectie van de primitieven in de gegeven bladknoop met de gegeven straal.     
    \item INTERSECTEER(volume, straal): Deze functie bepaalt de intersectie van het omhullend volume van de boom en de straal. Het resultaat bevat een booleaanse waarde, die aanduidt of het volume geraakt wordt door de straal. Als het volume geraakt wordt, bevat het ook de waarde tMin, de t-waarde waarvoor de straal het volume binnengaat, en tMax, de t-waarde waarvoor de straal het volume verlaat.
\end{enumerate}

De eerste functie is afhankelijk van de specifieke representatie voor bladknopen / inwendige knopen die gebruikt worden voor het specifieke type $\symBSP$ boom.
Bij de bespreking van de knooprepresentaties in de volgende secties, wordt steeds de implementatie van deze functie voor die knooprepresentatie getoond.
De tweede functie is hetzelfde voor alle knopen die besproken worden, elke driehoek in de bladknoop wordt op intersectie met de straal getest.
De derde functie bepaalt de intersectie van het omhullend volume van de volledige $\symBSP$ boom (= omhullend volume wortelknoop) met de straal. 
Bij alle besproken bomen wordt dit voorgesteld door een asgealigneerde balk.
Intersectie berekenen met een asgealigneerde balk is triviaal.

\section{$\symKd$ boom}
\label{sec:h4-kd}

De implementatie van de $\symKd$ boom is gebaseerd op de $\symKd$ boom implementatie van pbrt.
De $\symKd$ boom maakt gebruik van $\symKd$ knopen. 
Tabel \ref{tab:voorstelling-kd-knoop} toont de representatie van zowel inwendige $\symKd$ knopen als blad $\symKd$ knopen. 
Zowel inwendige knopen als bladknopen worden voorgesteld met 64 bits en bevatten de \textit{flags} variabele.
Als de \textit{flags} variabele gelijk is aan 3, is het een bladknoop.
Bij inwendige knopen stelt die variabele de as voor waarlangs gesplitst wordt: 0 is x, 1 is y en 2 is z.\\

De inwendige knopen bevatten extra informatie over hun splitsingsvlak via \textit{tSplit}.
Deze \textit{tSplit} variabele bepaalt de locatie van het vlak langs de as.
De knopen van een boom worden opgeslagen in een lijst, bij een inwendige knoop wordt zijn linkerkindknoop opgeslagen op de volgende index in de lijst.
De index van de rechterkindknoop, wordt opgeslagen in de \textit{tweedeKindIndex} variabele.\\

De bladknopen bevatten informatie over hun primitieven via \textit{n} en \textit{primitiefOffset}.
De \textit{n} variabele stelt het aantal primitieven in de bladknoop voor.
De \textit{primitiefOffset} variabele verwijst naar de primitieven in de bladknoop.
Als er maar één primitief in de knoop zit, wijst de variabele rechtstreeks naar het primitief.
Als er meerdere primitieven in de knoop zitten, stelt het een index voor in een lijst.
Elk element in die lijst met een index tussen \textit{primitiefOffset} en $\textit{primitiefOffset} + \textit{n}$ wijst dan naar een primitief in de bladknoop.\\

\begin{table}
    \centering
    \begin{tabular}{@{}|c|c|c|c|@{}} \toprule      
    Inwendig & bits & Blad & bits \\ \midrule
    tSplit & 32 & primitiefOffset & 32 \\
    flags  & 2  &  flags   & 2    \\
    tweedeKindIndex & 30 & $\symNbPrimitives$ & 30 \\ \hline \hline
    & 64 & & 64    \\ \bottomrule
    \end{tabular}
    \caption[Voorstelling $\symKd$ knoop]{Voorstelling $\symKd$ knoop - \small Deze tabel toont de nodige variabelen voor zowel inwendige als blad $\symKd$ knopen.}
    \label{tab:voorstelling-kd-knoop}
\end{table}            
\begin{dutchalgorithm}
    \begin{algorithmic}       
        \Function{doorkruis\_inwendige\_knoop}{$\symKd$ Knoop k, Straal s}
            \State $as \gets k_{flags}$
            \State $tVlak \gets $ \Call{kd\_vlak\_afstand}{$k_{tSplit}$, s, $\frac{1}{\vec{s_d}}$, as}
            \State $linksEerst \gets \vec{s_o}[as] < k_{tSplit}$ \Or $(\vec{s_o}[as] = k_{tSplit}$ \And $\vec{s_d}[as] \leq 0)$
            \State \Return $tVlak$, $linksEerst$
        \EndFunction
    \end{algorithmic}
    \caption{Doorkruisen van een inwendige $\symKd$ knoop.}
    \label{alg:kd-inwendige-doorkruising}
\end{dutchalgorithm}
\begin{dutchalgorithm}
    \begin{algorithmic}       
        \Function{kd\_vlak\_afstand}{splitPositie, s, inverseRichting, as}
            \State \Return $(splitPositie - \vec{s_o}[as]) * \vec{inverseRichting}[as]$
        \EndFunction
    \end{algorithmic}
    \caption{Intersectie tussen een asgealigneerd vlak en een straal.}
    \label{alg:kd-vlak-intersectie}
\end{dutchalgorithm}
Algoritme \ref{alg:kd-inwendige-doorkruising} toont het algoritme om inwendige $\symKd$ knopen te doorkruisen.
Het gebruikt de functie KD\_VLAK\_INTERSECTIE (algoritme \ref{alg:kd-vlak-intersectie}) om te intersecteren met een asgealigneerd vlak.
De inverse van de richting van de straal ($\frac{1}{\vec{s_d}}$) moet voor elke straal slechts één keer berekend worden.
Daarnaast bepaalt het algoritme aan de hand van de oorsprong en de richting van de straal of deze straal eerst door de linkse of rechtste kindknoop gaat.
\\

Algoritme \ref{alg:kd-beste-split} toont het algoritme om het beste splitsingsvlak te bepalen bij een $\symKd$ boom.
Het gebruikt de SAH functie die de SAH kost berekent voor het splitsingsvlak.
Deze functie gebruikt de oppervlaktes $\symSA_{rechts}$ en $\symSA_{links}$ van de gesplitste volumes en de aantallen primitieven $n_{links}$ en $n_{rechts}$ in de nieuwe knopen samen met de oppervlakte $bouwKnoop_{\symSA}$ van de huidige knoop.
Het splitsen in kindvolumes en het berekenen van deze oppervlaktes is triviaal omdat de omhullende volumes asgealigneerde balken zijn.
De waarden voor $n_{links}$ en $n_{rechts}$ worden efficiënt berekend door te sweepen.

\begin{dutchalgorithm}
    \begin{algorithmic}       
        \Function{bepaal\_beste\_splitsing}{bouwKnoop}
            \State $besteAs$, $besteT$, $besteKost$ $\gets None$, $None$, $\infty$
            \For{$as \in \{0, 1, 2\}$}
                \State $\forall i: randen[2i]\gets$ \Call{linkerrand}{$b_{primitieven}[i]$, as}
                \State $\forall i: randen[2i+1]\gets$ \Call{rechterrand}{$b_{primitieven}[i]$, as}
                \State \Call{sorteer}{randen}
                \State $n_{links}, n_{rechts} \gets 0, bouwKnoop_n$
                \For{$rand \in randen$}
                    \If{$rand_{isRechts}$}
                        \State $n_{rechts} \gets n_{rechts} - 1$
                    \EndIf
                    \State $\symSA_{links}$, $\symSA_{rechts} \gets $ \Call{split}{$bouwKnoop_{volume}$, as, $rand_t$}
                    \State $kost \gets$ \Call{SAH}{$\symCostTraversal$,$\symCost_\symIntersection$,$\symSA_{links}$,
                    $\symSA_{rechts}$,
                    $n_{links}$, $n_{rechts}$, $bouwKnoop_{\symSA}$}
                    \If{$kost < besteKost$}
                        \State $besteKost, besteAs, besteT \gets kost, as, rand_t$
                    \EndIf
                    \If{$rand_{isLinks}$}
                        \State $n_{links} \gets n_{links} + 1$
                    \EndIf
                \EndFor                
            \EndFor
            \State \Return $besteAs, besteT, besteKost$
        \EndFunction
    \end{algorithmic}
    \caption{Beste splitsing voor een bouwknoop b bij een $\symKd$ boom.}
    \label{alg:kd-beste-split}
\end{dutchalgorithm}



\section{$\symRBSP$ boom}
\label{sec:h4-rbsp}
De implementatie van de $\symRBSP$ boom is gebaseerd op de papers van \authorKammaje{} \cite{Kammaje} en \authorBudge{} \cite{Budge}.
De $\symRBSP$ boom maakt gebruik van $\symRBSP$ knopen. 
Tabel \ref{tab:voorstelling-rbsp-knoop} toont de representatie van zowel inwendige $\symRBSP$ knopen als blad $\symRBSP$ knopen.
Deze representatie heeft dezelfde variabelen en totale grootte als de representatie van \authorKammaje{} \cite{Kammaje}, maar het aantal bits per variabele verschilt.
Net als bij de $\symKd$ knoop worden zowel inwendige $\symRBSP$ knopen als blad $\symRBSP$ knopen voorgesteld met 64 bits en bevatten ze beide de \textit{flags} variabele.
Als de \textit{flags} variabele gelijk is aan $k$ (het aantal richtingen), is het een bladknoop.
Bij inwendige knopen stelt die variabele de index voor van de richting waarlangs gesplitst wordt.
De \textit{flags} variabele heeft hierdoor $\lceil log_2(k+1) \rceil$ bits nodig.
Om te zorgen dat het totaal aantal bits 64 blijft, wordt het aantal bits voor de \textit{tweedeKindIndex} variabele bij de inwendige $\symRBSP$ knoop en de \textit{n}  variable bij de blad $\symRBSP$ knoop verlaagt tot $32 - \lceil log_2(k+1) \rceil$.
Dit impliceert dat de $\symRBSP$ boom $2^{(\lceil log_2(k+1) \rceil - 2)}$ keer minder knopen kan bevatten.
Het maximaal aantal primitieven in een bladknoop daalt met dezelfde factor.
Voor het aantal primitieven vormt dit geen probleem, aangezien bladknopen steeds kleine aantallen primitieven bevatten.
Voor een $\symRBSP$ boom met $k = 13$ richtingen (maximaal $2^{28}$ knopen), betekent dit dat de boom 4 keer minder knopen kan bevatten dan de $\symKd$ boom (maximaal $2^{30}$ knopen). \\
\begin{table}
        \centering
        \begin{tabular}{@{}|c|c|c|c|@{}} \toprule      
        Inwendig & bits & Blad & bits \\ \midrule
        tSplit & 32 & primitiefOffset & 32 \\
        flags  & $\lceil log_2(k+1) \rceil$  &  flags   & $\lceil log_2(k+1) \rceil$   \\
        tweedeKindIndex & $32 - \lceil log_2(k+1) \rceil$ & $\symNbPrimitives$ &  $32 - \lceil log_2(k+1) \rceil$ \\ \hline \hline
        & 64 & & 64    \\ \bottomrule
        \end{tabular}
    \caption[Voorstelling $\symRBSP$ knoop]{Voorstelling $\symRBSP$ knoop - \small Deze tabel toont de nodige variabelen voor zowel inwendige als blad $\symRBSP$ knopen.}
    \label{tab:voorstelling-rbsp-knoop}    
\end{table}   

Algoritme \ref{alg:rbsp-inwendige-doorkruising} toont het algoritme om inwendige $\symRBSP$ knopen te doorkruisen.
Het gebruikt de functie VLAK\_INTERSECTIE (algoritme \ref{alg:vlak-intersectie}) om te intersecteren met een willekeurig geörienteerd vlak.
In vergelijking met de intersectie van een asgealigneerd vlak (algoritme \ref{alg:kd-vlak-intersectie}), moeten twee extra scalaire producten en een deling uitgerekend worden.
Aangezien, bij een $\symRBSP$ boom, het aantal richtingen waarmee deze scalaire producten berekend moeten worden, beperkt is, kunnen deze per straal éénmalig op voorhand berekend worden.
Deze techniek is toegepast door \authorBudge{} \cite{Budge}.
Omdat het effect hiervan, zeker bij stijgende $k$-waarden, niet altijd positief was, is dit niet toegepast.\\
% TODO
\begin{dutchalgorithm}
    \begin{algorithmic}       
        \Function{doorkruis\_inwendige\_knoop}{$\symRBSP$ Knoop k, Straal s}
            \State $richtingId \gets k_{flags}$
            \State $tVlak, projOorsprong, invProjRichting \gets $ \Call{vlak\_afstand}{$\vec{richtingen[richtingId]}$, $k_{tSplit}$, s}
            \State $linksEerst \gets projOorsprong < k_{tSplit}$ \Or $(projOorsprong = k_{tSplit}$ \And $invProjRichting \leq 0)$
            \State \Return $tVlak$, $linksEerst$
        \EndFunction
    \end{algorithmic}
    \caption{Doorkruisen van een inwendige $\symRBSP$ knoop.}
    \label{alg:rbsp-inwendige-doorkruising}
\end{dutchalgorithm}

\begin{dutchalgorithm}
    \begin{algorithmic}       
        \Function{vlak\_afstand}{$\vec{normaal}$, splitPositie, s}
            \State $projOorsprong \gets \vec{normaal} \cdot \vec{s_o}$
            \State $invProjRichting \gets \frac{1}{\vec{normaal} \cdot \vec{s_d}}$
            \State $afstand \gets (splitPos - projOorsprong) * invProjRichting $
            \State \Return afstand, projOorsprong, invProjRichting
        \EndFunction
    \end{algorithmic}
    \caption{Intersectie tussen een vlak en een straal.}
    \label{alg:vlak-intersectie}
\end{dutchalgorithm}

Het algoritme om het beste splitsingsvlak te bepalen bij een $\symRBSP$ boom is zeer gelijkaardig aan het algoritme voor de $\symKd$ boom.
In plaats van te sweepen over de drie assen, wordt er gesweept over alle $k$ richtingen.
Het bepalen van de linker- en rechterrand van de primitieven langs de as, vereist scalaire producten in tegenstelling tot bij de $\symKd$ boom.
Het splitsen van de omhullende volumes en het berekenen van de oppervlaktes is complexer omdat met \symKDOP{}s gewerkt moet worden in plaats van met asgealigneerde balken.
\authorBudge{} \cite{Budge} ontwikkelden een methode om deze splitsingsen en oppervlaktes incrementeel te berekenen tijdens het sweepen, deze methode wordt niet gebruikt.

\section{$\symRBSPKd$ boom}
\label{sec:h4-rbspkd}
De $\symRBSPKd$ boom is een nieuw concept dat in dit werk voor het eerst wordt voorgesteld.
Het combineert het concept van de $\symRBSP$ boom met de snelle $\symKd$ knoopdoorkruistechniek van \authorIze{} \cite{ize}.
De $\symRBSPKd$ knoop is identiek aan de $\symRBSP$ knoop in termen van representatie (tabel \ref{tab:voorstelling-rbsp-knoop}), maar de doorkruismethode is aangepast om $\symKd$ knopen sneller te kunnen doorkruisen.
Algoritme \ref{alg:rbspkd-inwendige-doorkruising} toont de aangepaste versie van het doorkruisalgoritme.\\

\begin{dutchalgorithm}
    \begin{algorithmic}       
        \Function{doorkruis\_inwendige\_knoop}{$\symRBSPKd$ Knoop k, Straal s}
            \State $richtingId \gets k_{flags}$
            \If {$richtingId < 3$}
                \State $tVlak \gets $ \Call{kd\_vlak\_afstand}{$k_{tSplit}$, s, $\frac{1}{\vec{s_d}}$, richtingId}
                \State $linksEerst \gets \vec{s_o}[richtingId] < k_{tSplit}$ \Or $(\vec{s_o}[richtingId] = k_{tSplit}$ \And $\vec{s_d}[richtingId] \leq 0)$
                \State \Return $tVlak$, $linksEerst$
            \Else
                \State $tVlak, projOorsprong, invProjRichting \gets $ \Call{vlak\_afstand}{$\vec{richtingen[richtingId]}$, $k_{tSplit}$, s}
                 \State $linksEerst \gets projOorsprong < k_{tSplit}$ \Or $(projOorsprong = k_{tSplit}$ \And $invProjRichting \leq 0)$
                \State \Return $tVlak$, $linksEerst$
            \EndIf
        \EndFunction
    \end{algorithmic}
    \caption{Doorkruisen van een inwendige $\symRBSPKd$ knoop.}
    \label{alg:rbspkd-inwendige-doorkruising}
\end{dutchalgorithm}

Het algoritme om het beste splitsingsvlak te bepalen bij een $\symRBSPKd$ boom wordt getoond in algoritme \ref{alg:rbspkd-beste-split}.
De sweeping over de $k$ richtingen wordt opgesplitst in twee delen: de sweeping over de $\symKd$ richtingen en de sweeping over de $\symBSP$ richtingen.
Het eerste deel berekent de $\symSAH$ kost met $\symCostTraversalKd$ als doorkruiskost.
Het tweede deel berekent twee $\symSAH$ kosten, één met een $\symCostTraversalBSP$ kost die lineair afhankelijk is van het aantal driehoeken en één met een vaste $\symCostTraversalBSP$ kost.
Deze eerste kost wordt rechtstreeks vergeleken met de beste kost van de $\symKd$ richtingen.
De tweede kost wordt enkel gebruikt als er geen voordelig $\symKd$ of $\symBSP$ splitsingsvlak gevonden wordt via de eerste kost.

\begin{dutchalgorithm}
    \begin{algorithmic}       
        \Function{bepaal\_beste\_splitsing}{bouwKnoop}
            \State $besteAs$, $besteT$, $besteKost$ $\gets None$, $None$, $\infty$
            \State $besteAsVast$, $besteTVast$, $besteKostVast$ $\gets None$, $None$, $\infty$
            \For{$as \in \{0, 1, 2\}$}
                \State Sorteer eindpunten langs de as
                \For{elk vlak langs de as}
                    \State Bereken de $\symSAH$ kost met $\symCostTraversalKd$ als doorkruiskost
                    \State Update $besteAs$, $besteT$, $besteKost$ als dit splitsingsvlak beter is dan het huidige beste splitsingsvlak
                \EndFor        
            \EndFor
            \For{$as \in \{3, 4, ..., k - 1\}$}
                \State Sorteer eindpunten langs de as
                \For{elk vlak langs de as}
                    \State Bereken de $\symSAH$ kost met een $\symCostTraversalBSP$ die lineair afhankelijk is van het aantal primitieven als doorkruiskost
                    \State Update $besteAs$, $besteT$, $besteKost$ als dit splitsingsvlak beter is dan het huidige beste splitsingsvlak    
                    \State Bereken de $\symSAH$ kost met een vaste $\symCostTraversalBSP$ als doorkruiskost
                    \State Update $besteAsVast$, $besteTVast$, $besteKostVast$ als dit splitsingsvlak beter is dan het huidige beste splitsingsvlak met vaste $\symBSP$ doorkruiskost.   
                \EndFor                 
            \EndFor
            \If{$besteKost < bouwKnoop_{\symNbPrimitives}*\symCost_\symIntersection$}
                \State \Return $besteAs, besteT, besteKost$
            \EndIf
            \If{$besteKostVast < bouwKnoop_{\symNbPrimitives}*\symCost_\symIntersection$}
                \State \Return $besteAsVast, besteTVast, besteKostVast$
            \EndIf
            \State \Return $None, None, \infty$
        \EndFunction
    \end{algorithmic}
    \caption{Beste splitsing voor een bouwknoop b bij een $\symRBSPKd$ boom.}
    \label{alg:rbspkd-beste-split}
\end{dutchalgorithm}

\section{$\symBSPize$ boom}
\label{sec:h4-bspize}
De $\symBSPize$ boom maakt gebruik van algemene $\symBSP$ knopen. 
Tabel \ref{tab:voorstelling-bsp-knoop} toont de representatie van zowel inwendige $\symBSP$ knopen als blad $\symBSP$ knopen.
De representatie van de $\symBSP$ knopen is identiek aan de representatie van de paper van \authorIze{} \cite{ize}.
Net als bij de $\symKd$ en $\symRBSP$ knopen bevatten zowel inwendige $\symBSP$ knopen als blad $\symBSP$ knopen de \textit{flags} variabele.
De \textit{flags} variabele wordt bij de $\symBSP$ boom enkel gebruikt om aan te geven of het een inwendige knoop (waarde 0) of een bladknoop (waarde 1) is.
Om de oriëntatie van het splitsingsvlak voor te stellen, zijn drie vlottend komma getallen nodig.
De \textit{splitsRichting} variabele bij inwendige $\symBSP$ knopen stelt de normaal van het splitsingsvlak voor.
De nodige geheugenruimte voor de inwendige $\symBSP$ knopen en de blad $\symBSP$ knopen is verschillend, maar aangezien elke knoop - in de huidige implementatie - evenveel geheugen moet innemen, neemt elke knoop 160 bits geheugen in.
Dit is 2.5 keer meer dan de $\symKd$ en $\symRBSP$ knopen en toont één van de nadelen van algemene $\symBSP$ bomen.\\
\begin{table}
        \centering
        \begin{tabular}{@{}|c|c|c|c|@{}} \toprule      
            Inwendig & bits & Blad & bits \\ \midrule
            tSplit & 32 & primitiefOffset & 32 \\
            flags  & 1  &  flags   & 1    \\
            tweedeKindIndex & 31 & $\symNbPrimitives$ & 31 \\
            splitsRichting & 96 &  &  \\ \hline \hline
            & 160 & & 64    \\ \bottomrule
        \end{tabular}
    \caption[Voorstelling $\symBSP$ knoop]{Voorstelling $\symBSP$ knoop - \small Deze tabel toont de nodige variabelen voor zowel inwendige als blad $\symBSP$ knopen.}
    \label{tab:voorstelling-bsp-knoop}
\end{table}   

Algoritme \ref{alg:bsp-inwendige-doorkruising} toont het algoritme om inwendige $\symBSP$ knopen te doorkruisen.
Het gebruikt, net als de $\symRBSP$ knoopdoorkruising, de functie VLAK\_INTERSECTIE (algoritme \ref{alg:vlak-intersectie}) om te intersecteren met een willekeurig geörienteerd vlak.
  
\begin{dutchalgorithm}
    \begin{algorithmic}       
        \Function{doorkruis\_inwendige\_knoop}{$\symBSP$ Knoop k, Straal s}
            \State $tVlak, projOorsprong, invProjRichting \gets $ \Call{vlak\_afstand}{$\vec{k_{splitsRichting}}$, $k_{tSplit}$, s}
            \State $linksEerst \gets projOorsprong < k_{tSplit}$ \Or $(projOorsprong = k_{tSplit}$ \And $invProjRichting \leq 0)$
            \State \Return $tVlak$, $linksEerst$
        \EndFunction
    \end{algorithmic}
    \caption{Doorkruisen van een inwendige $\symBSP$ knoop.}
    \label{alg:bsp-inwendige-doorkruising}
\end{dutchalgorithm}

Het algoritme om het beste splitsingsvlak te bepalen bij de $\symBSPize$ boom wordt getoond in algoritme \ref{alg:bspize-beste-split}.
Voor de $\symKd$ richtingen kan sweeping gebruikt worden, om het efficiënt te berekenen.
De $\symBSP$ vlakken worden per driehoek bekeken en maken gebruik van een $\symBVH$ boom om efficiënt te berekenen hoeveel driehoeken links en rechts van het splitsingsvlak liggen.
\authorIze{} \cite{ize} gebruiken een $\symBVH$ boom met sferen als omhullende volumes, deze implementatie maakt gebruik van de bestaande $\symBVH$ implementatie van pbrt die werkt met asgealigneerde balken als omhullende volumes.
Bij de $\symBSPize$ boom is het splitsen van de omhullende volumes en het berekenen van de oppervlaktes complexer omdat met algemene convexe veelvlakken gewerkt moet worden in plaats van met asgealigneerde balken.
Elke van deze convexe veelvlakken kan beschouwd worden als een \symKDOP{} met andere richtingen en is ook op deze manier geïmplementeerd.
Voor elke splitsingsrichting wordt gekeken of een voorouder van de huidige knoop al gesplitst is volgens een richting die binnen een hoek van $0.5^{\circ}$ ligt. 
De gebruikte richtingen worden gecontroleerd in volgorde dat ze gebruikt zijn.
Als een vorige richting gevonden wordt, wordt er opnieuw gesplitst volgens deze vorige richting in plaats van de nieuwe richting.


\begin{dutchalgorithm}
    \begin{algorithmic}       
        \Function{bepaal\_beste\_splitsing}{bouwKnoop}
            \State $besteAs$, $besteT$, $besteKost$ $\gets None$, $None$, $\infty$
            \For{$as \in \{0, 1, 2\}$}
                \State Sorteer eindpunten langs de as
                \For{elk vlak langs de as}
                    \State Bereken de $\symSAH$ kost met $\symCostTraversalKd$ als doorkruiskost
                    \State Update $besteAs$, $besteT$, $besteKost$ als dit splitsingsvlak beter is dan het huidige beste splitsingsvlak
                \EndFor        
            \EndFor
            \State $bvh \gets BOUW\_BVH(b)$
            \For{$primitief \in bouwKnoop_{primitieven}$}
                \For{elk splitsingsvlak bij het primitief}
                    \State Bereken het aantal primitieven links en rechts van het splitsingvlak met behulp van de $\symBVH$ boom
                    \State Bereken de $\symSAH$ kost met een vaste $\symCostTraversalBSP$ 
                    \State Update $besteAs$, $besteT$, $besteKost$ als dit splitsingsvlak beter is dan het huidige beste splitsingsvlak.
                \EndFor       
            \EndFor
            \State \Return $bestAs, besteT, besteKost$
        \EndFunction
    \end{algorithmic}
    \caption{Beste splitsing voor een bouwknoop b bij een $\symBSPize$ boom.}
    \label{alg:bspize-beste-split}
\end{dutchalgorithm}


\section{$\symBSPizefastkd$ boom}
\label{sec:h4-bspizefastkd}
De $\symBSPizefastkd$ boom maakt gebruik van $\symBSPKd$ knopen. 
Deze $\symBSPKd$ knopen zijn identiek aan $\symBSP$ knopen, met uitzondering van het aantal bits gebruikt door de \textit{flags} en \textit{tweedeKindIndex} variabelen. 
De \textit{flags} variabele heeft drie bits nodig, de waarden 0, 1 en 2 stellen de x, y en z as voor, waarde 3 betekent dat het een bladknoop is en waarde 4 betekent dat het een algemene $\symBSP$ knoop is met willekeurig splitsingsvlak.
Om het aantal bits een veelvoud van 32 te laten blijven, wordt de grootte van de \textit{tweedeKindIndex} variabele met 2 bits verminderd.
Hierdoor kan de $\symBSPizefastkd$ boom vier keer minder knopen bevatten dan de $\symBSPize$ boom.
Tabel \ref{tab:voorstelling-bspkd-knoop} toont de representatie van zowel inwendige $\symBSPKd$ knopen als blad $\symBSPKd$ knopen.\\

\begin{table}
    \centering
    \begin{tabular}{@{}|c|c|c|c|@{}} \toprule      
        Inwendig & bits & Blad & bits \\ \midrule
        tSplit & 32 & primitiefOffset & 32 \\
        flags  & 3  &  flags   & 3   \\
        tweedeKindIndex & 29 & $\symNbPrimitives$ & 29 \\
        splitsRichting & 96 &  &  \\ \hline \hline
        & 160 & & 64    \\ \bottomrule
    \end{tabular}
\caption[Voorstelling $\symBSPKd$ knoop]{Voorstelling $\symBSPKd$ knoop - \small Deze tabel toont de nodige variabelen voor zowel inwendige als blad $\symBSPKd$ knopen.}
\label{tab:voorstelling-bspkd-knoop}
\end{table}   

Het grootste verschil tussen $\symBSP$ knopen en $\symBSPKd$ knopen is dat de $\symBSPKd$ knopen gebruik maken van de snellere $\symKd$ doorkruistechniek voor inwendige $\symKd$ knopen.
Algoritme \ref{alg:bspkd-inwendige-doorkruising} toont de aangepaste versie van het doorkruisalgoritme voor $\symBSPKd$ knopen.


\begin{dutchalgorithm}
    \begin{algorithmic}       
        \Function{doorkruis\_inwendige\_knoop}{$\symBSPKd$ Knoop k, Straal s}
            \State $isKdKnoop \gets k_{flags} < 4$
            \If {$isKdKnoop$}
                \State $as \gets k_{flags}$
                \State $tVlak \gets $ \Call{kd\_vlak\_afstand}{$k_{tSplit}$, s, $\frac{1}{\vec{s_d}}$, as}
                \State $linksEerst \gets \vec{s_o}[as] < k_{tSplit}$ \Or $(\vec{s_o}[as] = k_{tSplit}$ \And $\vec{s_d}[as] \leq 0)$
                \State \Return $tVlak$, $linksEerst$
            \Else
                \State $tVlak, projOorsprong, invProjRichting \gets $ \Call{vlak\_afstand}{$\vec{k_{splitsRichting}}$, $k_{tSplit}$, s}
                 \State $linksEerst \gets projOorsprong < k_{tSplit}$ \Or $(projOorsprong = k_{tSplit}$ \And $invProjRichting \leq 0)$
                \State \Return $tVlak$, $linksEerst$
            \EndIf
        \EndFunction
    \end{algorithmic}
    \caption{Doorkruisen van een inwendige $\symBSPKd$ knoop.}
    \label{alg:bspkd-inwendige-doorkruising}
\end{dutchalgorithm}

Het algoritme om het beste splitsingsvlak te bepalen bij de $\symBSPizefastkd$ boom wordt getoond in algoritme \ref{alg:bspizekd-beste-split} en is zeer gelijkaardig aan het algoritme voor de $\symBSPize$ boom.
Door de snelle $\symKd$ doorkruising is het nodig om aparte doorkruiskosten voor $\symKd$ en $\symBSP$ te gebruiken.
Zoals besproken in sectie \ref{sec:h2-bspizefastkd} wordt gebruik gemaakt van een $\symCostTraversalBSP$ die lineair afhankelijk is van het aantal driehoeken om $\symKd$ knopen te bevoordelen.
Als er geen nuttig $\symKd$ splitsingsvlak of $\symBSP$ splitsingsvlak met deze $\symCostTraversalBSP$ gevonden wordt, worden de $\symBSP$ splitsingsvlakken opnieuw bekeken met een vaste $\symCostTraversalBSP$.
\begin{dutchalgorithm}
    \begin{algorithmic}       
        \Function{bepaal\_beste\_splitsing}{bouwKnoop}
            \State $besteAs$, $besteT$, $besteKost$ $\gets None$, $None$, $\infty$
            \State $besteAsVast$, $besteTVast$, $besteKostVast$ $\gets None$, $None$, $\infty$
            \For{$as \in \{0, 1, 2\}$}
                \State Sorteer eindpunten langs de as
                \For{elk vlak langs de as}
                    \State Bereken de $\symSAH$ kost met $\symCostTraversalKd$ als doorkruiskost
                    \State Update $besteAs$, $besteT$, $besteKost$ als dit splitsingsvlak beter is dan het huidige beste splitsingsvlak
                \EndFor        
            \EndFor
            \State $bvh \gets BOUW\_BVH(b)$
            \For{$primitief \in bouwKnoop_{primitieven}$}
                \For{elk splitsingsvlak bij het primitief}
                    \State Bereken het aantal primitieven links en rechts van het splitsingvlak met behulp van de $\symBVH$ boom
                    \State Bereken de $\symSAH$ kost met een $\symCostTraversalBSP$ die lineair afhankelijk is van het aantal primitieven als doorkruiskost 
                    \State Update $besteAs$, $besteT$, $besteKost$ als dit splitsingsvlak beter is dan het huidige beste splitsingsvlak    
                    \State Bereken de $\symSAH$ kost met een vaste $\symCostTraversalBSP$ als doorkruiskost
                    \State Update $besteAsVast$, $besteTVast$, $besteKostVast$ als dit splitsingsvlak beter is dan het huidige beste splitsingsvlak met vaste $\symBSP$ doorkruiskost.             
                \EndFor       
            \EndFor
            \If{$besteKost < bouwKnoop_{\symNbPrimitives}*\symCost_\symIntersection$}
                \State \Return $besteAs, besteT, besteKost$
            \EndIf
            \If{$besteKostVast < bouwKnoop_{\symNbPrimitives}*\symCost_\symIntersection$}
                \State \Return $besteAsVast, besteTVast, besteKostVast$
            \EndIf
            \State \Return $None, None, \infty$
        \EndFunction
    \end{algorithmic}
    \caption{Beste splitsing voor een bouwknoop b bij een $\symBSPizefastkd$ boom.}
    \label{alg:bspizekd-beste-split}
\end{dutchalgorithm}


\section{$\symBSPsweep$ boom}
\label{sec:h4-bspsweep}
De $\symBSPsweepmaybewithkd$ en $\symBSPsweepfastkd$ bomen maken gebruik van respectievelijk de $\symBSP$ en $\symBSPKd$ knopen.
Hierdoor zijn de doorkruismethodes voor inwendige knopen identiek aan die gebruikt bij de $\symBSPize$ en $\symBSPizefastkd$ bomen.
Algoritme \ref{alg:bspsweep-beste-split} toont het algoritme om het beste splitsingsvlak te bepalen bij een $\symBSPsweep$ boom.
Dit algoritme gebruikt de BEPAAL\_RICHTINGEN functie die de specifieke soort $\symBSPsweep$ boom bepaalt.
Zoals de naam van de boom suggereert wordt in elke knoop langs elk splitsingsrichting gesweept om het beste splitsingsvlak op een efficiënte manier te vinden.
Hierdoor is er geen nood aan een hulpstructuur zoals bij de $\symBSPize$ boom.

\begin{dutchalgorithm}
    \begin{algorithmic}       
        \Function{bepaal\_beste\_splitsing}{bouwKnoop}
            \State $besteAs$, $besteT$, $besteKost$ $\gets None$, $None$, $\infty$
            \State $richtingen \gets $ \Call{bepaal\_richtingen}{bouwKnoop, $k$}
            \For{$richting \in richtingen$}
                \State Sorteer eindpunten volgens de richting
                \For{elk vlak volgens de richting}
                    \State Bereken de $\symSAH$ kost met $\symCostTraversalKd$ of $\symCostTraversalBSP$ als doorkruiskost
                    \State Update $besteAs$, $besteT$, $besteKost$ als dit splitsingsvlak beter is dan het huidige beste splitsingsvlak
                \EndFor     
            \EndFor
            \State \Return $bestAs, besteT, besteKost$
        \EndFunction
    \end{algorithmic}
    \caption{Beste splitsing voor een bouwknoop b bij een $\symBSPsweepmaybewithkd$  boom.}
    \label{alg:bspsweep-beste-split}
\end{dutchalgorithm}

\paragraph{$\mathbf{\symBSPrandom}$ boom}
De $\symBSPrandom$ boom kiest op elk niveau $k$ random richtingen.
Algoritme \ref{alg:bsprandom-richtingen} toont de implementatie van de BEPAAL\_RICHTINGEN functie voor de $\symBSPrandom$ boom. 
Deze functie gebruikt de UPDATE\_ZIN functie die zorgt dat de zin van de richting zo gekozen wordt dat de eerste niet-nul component van de vector positief is. Als de x- en y-component bijvoorbeeld beiden nul zijn, wordt gezorgd dat de z-component positief is.
Dit is nodig omdat de vlakken langs een richting identiek zijn aan de vlakken langs zijn omgekeerde richting.

\begin{dutchalgorithm}
    \begin{algorithmic}       
        \Function{bepaal\_richtingen}{b, $k$}
            \State $richtingen \gets \emptyset$
            \For{$i \in \{0,1,2,...,k-1\}$}
                    \State $\phi \gets $ \Call{uniform}{0, $2\pi$}
                    \State $\theta \gets acos($\Call{uniform}{-1, 1}$)$
                    \State $\vec{richting} \gets \{ sin(\theta)cos(\phi), sin(\theta)sin(\phi), cos(\theta) \}$
                    \State $\vec{richting} \gets $ \Call{update\_zin}{$\vec{richting}$}
                    \State \Call{add}{richtingen, $\vec{richting}$}
            \EndFor
            \State \Return $richtingen$
        \EndFunction
    \end{algorithmic}
    \caption{Generatie richtingen voor de $\symBSPrandom$  boom.}
    \label{alg:bsprandom-richtingen}
\end{dutchalgorithm}

\paragraph{$\mathbf{\symBSParbitrary}$ boom}
De $\symBSParbitrary$ boom kiest op elk niveau de normalen van $k$ driehoeken als richtingen.
Als er minder dan $k$ driehoeken zijn, worden de normalen van alle driehoeken gebruikt.
Algoritme \ref{alg:bsparbitrary-richtingen} toont de implementatie van de BEPAAL\_RICHTINGEN functie voor de $\symBSParbitrary$ boom.

\begin{dutchalgorithm}
    \begin{algorithmic}       
        \Function{bepaal\_richtingen}{b, $k$}
            \State $richtingen \gets \emptyset$
            \While{$richtingen_{lengte} < $ \Call{min}{$k-1$,$b_n - 1$}}
                    \State $\vec{richting} \gets \vec{b_{primitieven}[\Call{random\_int}{0, b_n}]_{normaal}}$
                    \State $\vec{richting} \gets $ \Call{update\_zin}{$\vec{richting}$}
                    \State \Call{add}{richtingen, $\vec{richting}$}
            \EndWhile
            \State \Return $richtingen$
        \EndFunction
    \end{algorithmic}
    \caption{Generatie richtingen voor de $\symBSParbitrary$  boom.}
    \label{alg:bsparbitrary-richtingen}
\end{dutchalgorithm}

\paragraph{$\mathbf{\symBSPcluster}$ boom}
De $\symBSPcluster$ boom kiest op elk niveau $k$ richtingen door de normalen te clusteren in $k$ groepen via K-means clustering.
De centra van de clusters zijn de gebruikte richtingen.
Algoritme \ref{alg:bspcluster-richtingen} toont de implementatie van de BEPAAL\_RICHTINGEN functie voor de $\symBSPcluster$ boom.
Als er tijdens het bepalen van de clusters, een cluster leeg wordt, worden alle clusters opnieuw willekeurig gegenereerd.

\begin{dutchalgorithm}
    \begin{algorithmic}       
        \Function{bepaal\_richtingen}{b, $k$}
            \State $\forall i \in \{0,1,...,b_n-1\}: \vec{normalen[i]} \gets $\Call{update\_zin}{$\vec{b_{primitieven}[i]_{normaal}}$}
            \If{$b_n \leq k$}
                \State \Return $normalen$
            \EndIf
            \State $clusterCentra \gets \emptyset$
            \While{$clusterCentra_{lengte} < k$}
                \State $\vec{clusterCentrum} \gets \vec{normalen[\Call{random\_int}{0, b_n}]}$
                \State \Call{add}{clusterCentra, $\vec{clusterCentrum}$}
            \EndWhile
            \State $iteratie \gets 0$
            \State $oudeClusterCentra \gets clusterCentra$
            \While{$iteratie < maxIteraties$ \And ($iteratie = 0$ \Or $oudeClusterCentra \neq clusterCentra$ )}
                \State $\forall i \in \{0,1..k-1\}: clusters[i] \gets \{\vec{n} | \vec{n} \in normalen, \neg\exists j: j \neq i, \angle(\vec{clusterCentra[i]}, \vec{n}) > \angle(\vec{clusterCentra[j]}, \vec{n})\} $
                \State $oudeClusterCentra \gets clusterCentra$
                \For{$i \in \{0,1,2..k-1\}$}
                    \If{$clusters[i] = \emptyset$}
                        \State $clusterCentra \gets \emptyset$
                        \While{$clusterCentra_{lengte} < k$}
                            \State $\vec{clusterCentrum} \gets \vec{normalen[\Call{random\_int}{0, b_n}]}$
                            \State \Call{add}{clusterCentra, $\vec{clusterCentrum}$}
                            \State $break$
                        \EndWhile
                    \EndIf
                    \State $\vec{clusterCentra[i]} \gets $ \Call{gemiddelde}{$clusters[i]$}
                \EndFor
                \State $iteratie \gets iteratie + 1$
            \EndWhile
            \State \Return $clusterCentra$
        \EndFunction
    \end{algorithmic}
    \caption{Generatie richtingen voor de $\symBSPcluster$  boom.}
    \label{alg:bspcluster-richtingen}
\end{dutchalgorithm}

De implementaties van de $\symBSPrandomkd$, $\symBSParbitrarykd$ en $\symBSPclusterkd$ bomen zijn bijna identiek aan de implementaties van de $\symBSPrandom$, $\symBSParbitrary$ en $\symBSPcluster$ bomen.
Het enige verschil is dat de $\symKd$ richtingen steeds gebruikt worden en dat er maar $k - 3$ richtingen gegenereert worden via de BEPAAL\_RICHTINGEN functies.
De implementaties van de $\symBSPrandomfastkd$, $\symBSParbitraryfastkd$ en $\symBSPclusterfastkd$ bomen hebben een aangepaste BEPAAL\_BESTE\_SPLITSING methode die de aangepaste $\symSA$ heuristiek van de $\symBSPizefastkd$ boom toepast.
Algoritme \ref{alg:bspsweepfastkd-beste-split} toont deze implementatie.

\begin{dutchalgorithm}
    \begin{algorithmic}       
        \Function{bepaal\_beste\_splitsing}{bouwKnoop}
            \State $besteAs$, $besteT$, $besteKost$ $\gets None$, $None$, $\infty$
            \State $besteAsVast$, $besteTVast$, $besteKostVast$ $\gets None$, $None$, $\infty$
            \State $richtingen \gets $ \Call{bepaal\_richtingen}{b}
            \For{$richting \in richtingen$}
                \State Sorteer eindpunten volgens de richting
                \If{$richting_{isKdRichting}$}
                    \For{elk vlak volgens de richting}
                        \State Bereken de $\symSAH$ kost met $\symCostTraversalKd$ als doorkruiskost
                        \State Update $besteAs$, $besteT$, $besteKost$ als dit splitsingsvlak beter is dan het huidige beste splitsingsvlak
                    \EndFor   
                \Else
                    \For{elk vlak volgens de richting}
                        \State Bereken de $\symSAH$ kost met een $\symCostTraversalBSP$ die lineair afhankelijk is van het aantal primitieven als doorkruiskost 
                        \State Update $besteAs$, $besteT$, $besteKost$ als dit splitsingsvlak beter is dan het huidige beste splitsingsvlak 
                        \State Bereken de $\symSAH$ kost met een vaste $\symCostTraversalBSP$ als doorkruiskost
                        \State Update $besteAsVast$, $besteTVast$, $besteKostVast$ als dit splitsingsvlak beter is dan het huidige beste splitsingsvlak met vaste $\symBSP$ doorkruiskost. 
                \EndFor
                \EndIf   
            \EndFor
            \If{$besteKost < bouwKnoop_{\symNbPrimitives}*\symCost_\symIntersection$}
                \State \Return $besteAs, besteT, besteKost$
            \EndIf
            \If{$besteKostVast < bouwKnoop_{\symNbPrimitives}*\symCost_\symIntersection$}
                \State \Return $besteAsVast, besteTVast, besteKostVast$
            \EndIf
            \State \Return $None, None, \infty$
        \EndFunction
    \end{algorithmic}
    \caption{Beste splitsing voor een bouwknoop b bij een $\symBSPsweepfastkd$  boom.}
    \label{alg:bspsweepfastkd-beste-split}
\end{dutchalgorithm}

% \begin{table}
%     \centering
%     \begin{tabular}{@{}|c|c|@{}} \toprule      
%     $\symBSP$ boom & Knooptype \\ \midrule
%     $\symKd$ & $\symKd$ \\
%     $\symRBSP$ & $\symRBSP$  \\
%     $\symRBSPKd$ & $\symRBSPKd$  \\
%     $\symBSPize$ &  $\symBSP$ \\
%     $\symBSPizefastkd$ & $\symBSPKd$ \\
%     $\symBSPsweepmaybewithkd$ & $\symBSP$ \\
%     $\symBSPsweepfastkd$ & $\symBSPKd$ \\ \bottomrule
%     \end{tabular}
%     \caption[Gebruikte soorten knopen voor alle soorten $\symBSP$ bomen]{Gebruikte soorten knopen voor alle soorten $\symBSP$ bomen - \small}
%     \label{tab:voorstelling-knoop-boom}
% \end{table}

%
%\section{$\symKd$ boom}
%\section{$\symRBSP$ boom}
%\section{$\symBSPize$}
%\section{$\symBSPsweep$}
%\subsection{Algemeen}
%\subsection{$\symBSPrandom$}
%\subsection{$\symBSParbitrary$}
%\subsection{$\symBSPcluster$}


%%% Local Variables: 
%%% mode: latex
%%% TeX-master: "masterproef"
%%% End: 